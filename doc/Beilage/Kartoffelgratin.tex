\documentclass[MeineRezepte.tex]{subfiles}
\begin{document}
\begin{recipe}[
    preparationtime = {\unit[15]{Min}},
    bakingtime = {\unit[30]{Min}},
    bakingtemperature = {\unit[180]{\celsius}},
    portion = {\portion{4}},
    source = Lili
]{Kartoffelgratin}\index{Kartoffelgratin}
    \graph{
        big=Kartoffelgratin_0
    }
    \ingredients{
        8                   & Kartoffeln, groß\\
        {\unit[200]{ml}}    & Sahne\\
        {\unit[50]{ml}}     & Milch\\
        {\unit[2]{TL}}      & Brühenpulver\\
                            & Salz\\
                            & Pfeffer
    }
    \preparation{
        \step Die Kartoffeln schälen, waschen, hobeln und dachziegelartig auf ein Backblech auslegen.
        \step Jede Schicht mit etwas Salz, Pfeffer und Brühe würzen.
        \step Sahne über die Kartoffelscheiben gießen und mit Milch auffüllen, bis man zwischen ihnen Milch sieht.
        \step Im Backofen 30~Minuten bei {\unit[180]{\celsius}} backen, bis das Gratin leicht bräunt.
    }
\end{recipe}
\end{document}
