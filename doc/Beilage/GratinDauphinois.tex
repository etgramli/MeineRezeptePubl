\documentclass[MeineRezepte.tex]{subfiles}
\begin{document}
\begin{recipe}[
    preparationtime = {\unit[15]{Min}},
    bakingtime = {\unit[1]{h}},
    bakingtemperature = {\unit[180]{\celsius}},
    portion = {\portion{4}},
    source = Etienne
]{Gratin dauphinois}\index{Gratin dauphinois}
    \graph{
        big=GratinDauphinois_0,
        small=GratinDauphinois_1
    }
    \ingredients{
        8                   & Kartoffeln, groß\\
        {\unit[200]{ml}}    & Sahne\\
        {\unit[250]{ml}}    & Milch\\
        2                   & Knoblauchzehen\\
        {\unit[100]{g}}     & Gruyère\\
        {\unit[$\nicefrac{1}{2}$]{TL}}    & Salz\\
                            & Pfeffer\\
                            & Muskat
    }
    \preparation{
        \step Die Knoblauchzehen schälen, zerquetschen und in eine Auflaufform legen.
        \step Kartoffeln schälen, waschen, in dicke Scheiben schneiden und dachziegelartig in die Auflaufform schichten.
        \step Salzen, pfeffern und Muskat dazu reiben. Milch und Sahne dazu geben. Gruyère darauf verteilen.
        \step Im Backofen eine~Stunde bei {\unit[180]{\celsius}} backen, bis das Gratin leicht bräunt.
    }
\end{recipe}
\end{document}
