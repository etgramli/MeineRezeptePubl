\makeatletter
\def\input@path{{../../}}
\makeatother
\providecommand{\main}{../../}
\documentclass[MeineRezepte.tex]{subfiles}
\begin{document}
\begin{recipe}[
    preparationtime = {\unit[15]{Min}},
    bakingtime = {\unit[30]{Min}},
    portion = {\portion{3}},
    source = Lili
]{Kartoffelgulasch}\index{Kartoffelgulasch}\index{Gulasch!Kartoffelgulasch}
    \graph{
        small=Kartoffelgulasch_1,
        big=Kartoffelgulasch_0
    }
    \ingredients{
        6                                 & Kartoffeln, mittel\\
        {\unit[50]{g}}                    & Schinken\\
        1                                 & Zwiebel\\
        {\unit[1]{TL}}                    & Paprikapulver\\
        {\unit[1]{EL}}                    & Tomatenmark\\
        {\unit[$\nicefrac{1}{2}$]{EL}}    & Essig\\
        {\unit[1]{TL}}                    & Thymian\\
        {\unit[500]{ml}}                  & Brühe\\
        2                                 & Wienerle\\
        {\unit[100]{g}}                   & Sahne / Schmand\\
                                          & Öl\\
                                          & Pfeffer\\
        opt. 2                            & Rüben\\
        opt. 1                            & Paprika, rot
    }
    \preparation{
        \step Zwiebeln und Schinken fein hacken, in Öl andünsten und vom Herd nehmen.
        \step Paprika einrühren, Tomatenmark, Essig, Thymian dazu geben und kurz braten, dabei immer rühren.
        \step Kartoffeln schälen, grob würfeln, dazu geben und mit Brühe ablöschen.
        \step Topf zudecken und 25 Minuten schmoren lassen; Wienerle hinein schneiden, Sahne einrühren.
    }
    \hint{%
        Schmeckt sehr fein mit Rüble!\newline
        Geht im Schnellkochtopf auch!
    }
\end{recipe}
\end{document}
