\makeatletter
\def\input@path{{../../}}
\makeatother
\providecommand{\main}{../../}
\documentclass[MeineRezepte.tex]{subfiles}
\begin{document}
\begin{recipe}[
    preparationtime = {\unit[15]{Min}},
    portion = {\portion{2}},
    source = Etienne
]{Steak mit Wacholder-Sauce}\index{Steak mit Wacholder-Sauce}
    \graph{
        big=SteakMitWacholderSauce_0
    }
    \ingredients{
        2                     & Rindersteaks\\
        4                     & Champignons, groß\\
        1                     & Zwiebel, klein\\
        {\unit[150]{ml}}      & Rotwein\\
        {\unit[50]{ml}}       & Wasser\\
        2                     & Wacholderbeeren\\
        {\unit[1]{Prise}}     & Rosmarin\\
                              & Salz\\
                              & Pfeffer
    }
    \preparation{
        \step Champignons in dicke Scheiben und Zwiebeln in grobe Würfel schneiden und in Öl dünsten.
        \step Wenn sie bräunen, Champignons und Zwiebeln zur Seite schieben und Steaks braten.
        \step Die Steaks in Alufolie zur Seite stellen, und Wacholderbeeren mörsern und kurz anbraten.
        \step Champignons und Zwiebeln mit Rotwein und Wasser ablöschen, würzen und kurz köcheln lassen.
        \step Die Steaks mit der Sauce servieren.
    }
\end{recipe}
\end{document}
