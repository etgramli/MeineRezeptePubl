\makeatletter
\def\input@path{{../../}}
\makeatother
\providecommand{\main}{../../}
\documentclass[MeineRezepte.tex]{subfiles}
\begin{document}
\begin{recipe}[
    preparationtime = {\unit[20]{Min}},
    portion = {\portion{4}},
    source = Lili
]{Spaghetti mit Zucchini und Auberginen}\index{Spaghetti mit Zucchini und Auberginen}
    \graph{
        small=SpaghettiMitZucchiniUndAubergine_1,
        big=SpaghettiMitZucchiniUndAubergine_0
    }
    \ingredients{
        {\unit[600]{g}}    & Spaghetti\\
        1                  & Zucchini, klein\\
        1                  & Aubergine, klein\\
        1                  & Knoblauchzehe\\
                           & Olivenöl\\
                           & Weißwein\\
                           & Salz\\
                           & Pfeffer\\
                           & Basilikum
    }
    \preparation{
        \step Die Zucchini und Aubergine waschen, abtrocknen und würfeln. Währenddessen Spaghetti kochen.
        \step Knobi klein würfeln, im Olivenöl dünsten, dann Zucchini und Aubergine anbraten.
        \step Mit Salz, Pfeffer und Basilikum würzen, mit Schuss Weißwein 10-20 Minuten schmoren.
        \step Die Spaghetti mit der Sauce mischen und servieren.
    }
    \hint{%
        Um die Aubergine weniger bitter zu machen kann man sie salzen, ziehen lassen und dann abbrausen.\newline
        War zu salzig, da wir die Auberginen nicht abgebraust haben
    }
\end{recipe}
\end{document}
