\makeatletter
\def\input@path{{../../}}
\makeatother
\providecommand{\main}{../../}
\documentclass[MeineRezepte.tex]{subfiles}
\begin{document}
\begin{recipe}[
    preparationtime = {\unit[45]{Min}},
    bakingtime = {\unit[100]{Min}},
    portion = {\portion{4}},
    source = Etienne
]{Hirschgulasch}\index{Hirschgulasch}\index{Gulasch!Hirschgulasch}
    \graph{
        big=Hirschgulasch_0
    }
    \ingredients{
        {\unit[1]{kg}}        & Gulasch vom Wild\\
        2                     & Zwiebeln\\
        1                     & Lorbeerblatt\\
        2                     & Nelken\\
        4                     & Wacholderbeeren\\
        {\unit[4]{Zweige}}    & Thymian\\
        {\unit[2]{Zweige}}    & Rosmarin\\
        {\unit[250]{ml}}      & Rotwein\\
        {\unit[4]{EL}}        & Obstessig\\
        {\unit[2]{EL}}        & Butterschmalz\\
        {\unit[1]{EL}}        & Brühe\\
        {\unit[2]{EL}}        & Tomatenmark\\
        {\unit[1]{TL}}        & Pfeffer, grün\\
        {\unit[250]{g}}       & Champignons\\
        {\unit[150]{g}}       & Saure Sahne\\
                              & Salz, Pfeffer
    }
    \preparation{
        \step Eine Zwiebel würfeln und mit Gulasch, Lorbeer, Nelken, Wacholder und Kräutern in ein Gefäß legen.
        \step Wein und Essig dazu tun und 2 bis 3 Tage marinieren lassen.
        \step Fleisch abtropfen lassen, die Marinade durch ein Sieb abseihen und zur Seite stellen.
        \step Eine Zwiebel würfeln und mit dem Fleisch in Butter scharf anbraten; Tomatenmark kurz bräunen.
        \step Mit Marinade ablöschen, grünen Pfeffer und Brühe dazu geben und 70~Minuten köcheln lassen.
        \step Champignons schneiden, dazu geben, eventuell noch Wein nachgießen und noch 30~Minuten köcheln.
        \step Mit Salz, Pfeffer abschmecken, abkühlen lassen und mit saurer Sahne servieren.
    }
\end{recipe}
\end{document}
