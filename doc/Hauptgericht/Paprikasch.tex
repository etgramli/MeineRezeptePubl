\makeatletter
\def\input@path{{../../}}
\makeatother
\providecommand{\main}{../../}
\documentclass[MeineRezepte.tex]{subfiles}
\begin{document}
\begin{recipe}[
    preparationtime = {\unit[30]{Min}},
    bakingtime = {\unit[45]{Min}},
    portion = {\portion{4}},
    source = Lili
]{Paprikasch}\index{Paprikasch}
    \graph{
        big=Paprikasch_0
    }
    \ingredients{
        1                   & Hähnchen\\
        {\unit[250]{ml}}    & Brühe\\
        2                   & Zwiebeln\\
        3                   & Paprika, rot\\
        {\unit[2]{EL}}      & Öl\\
        {\unit[1]{EL}}      & Tomatenmark\\
        {\unit[1]{EL}}      & Paprikapulver\\
                            & Salz\\
                            & Pfeffer
    }
    \preparation{
        \step Hähnchen waschen, trocken tupfen, in Teile schneiden, mit Salz, Pfeffer und Paprikapulver einreiben.
        \step Paprika und Zwiebel würfeln und in heißem Öl anbraten.
        \step Hähnchenteile dazu geben und unter Rühren andünsten.
        \step Mit Brühe aufgießen, Deckel schließen und 35-45~Minuten köcheln lassen, gelegentlich umrühren.
    }
    \hint{%
        Dazu passen Baguette-Brot, Salzkartoffeln und Salat.\newline
        Man kann das Hähnchen auch häuten. Schmeckt lecker!!
    }
\end{recipe}
\end{document}
