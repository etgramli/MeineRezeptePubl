\documentclass[MeineRezepte.tex]{subfiles}
\begin{document}
\begin{recipe}[
    preparationtime = {\unit[40]{Min}},
    bakingtime = {\unit[30]{Min}},
    portion = {\portion{4}},
    source = Etienne
]{Pilzragout}\index{Pilzragout}
    \graph{
        big=Pilzragout_0
    }
    \ingredients{
        {\unit[200]{g}}     & Champignons\\
        1                   & Zwiebel\\
        {\unit[80]{g}}      & Rote Linsen\\
        {\unit[2]{EL}}      & Tomatenmark\\
        1                   & Lorbeerblatt\\
        {\unit[20]{g}}      & Sellerie\\
        {\unit[400]{ml}}    & Brühe\\
        1                   & Rübe\\
                            & Pfeffer\\
                            & Salz\\
                            & Kräuter
    }
    \preparation{
        \step Die Pilze putzen, in grobe Würfel schneiden, in Olivenöl anbraten und zur Seite stellen.
        \step Zwiebel in kleine Würfel schneiden und mit den roten Linsen in Olivenöl andünsten.
        \step Tomatenmark dazu geben und kurz rösten; mit Brühe ablöschen und mit dem Lorbeer 20~Minuten köcheln.
        \step Rübe in Scheiben schneiden, mit Pilzen und Sellerie dazu geben und nochmal 10 Minuten köcheln.
        \step Mit Salz, Pfeffer und Kräutern abschmecken und servieren.
    }
    \hint{%
        Dazu passen Semmelknödel sehr gut.
    }
\end{recipe}
\end{document}
