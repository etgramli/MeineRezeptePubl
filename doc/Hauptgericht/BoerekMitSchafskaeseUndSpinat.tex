\makeatletter
\def\input@path{{../../}}
\makeatother
\providecommand{\main}{../../}
\documentclass[MeineRezepte.tex]{subfiles}
\begin{document}
\begin{recipe}[
    bakingtime = {\unit[25-30]{Min}},
    bakingtemperature = {\unit[180]{\celsius}},
    portion = {\portion{4}},
    source = Lili
]{Börek mit Schafskäse und Spinat}\index{Boerek mit Schafskaese und Spinat@Börek mit Schafskäse und Spinat}
    \graph{
        big=Boerek_0
    }
    \ingredients{
        {\unit[1]{P.}}     & Yufka-Teig\\
        {\unit[200]{g}}    & Creme Fraîche\\
        1                  & Ei\\
        {\unit[200]{g}}    & Schafskäse\\
                           & Käse, gerieben\\
        {\unit[250]{g}}    & Spinat\\
        1                  & Zwiebel, klein\\
        1                  & Knoblauchzehe\\
                           & Milch\\
                           & Sahne\\
                           & Sesamkörner\\
                           & Kümmel\\
                           & Pfeffer
    }
    \preparation{
        \step Zwiebel und Knoblauch anschwitzen und mit Spinat und Sahne vermengen, mit Salz und Pfeffer würzen.
        \step Creme Fraîche mit dem Ei verquirlen und pfeffern; Backblech gut buttern.
        \step Zwei Blätter Teig aufs Blech legen, die Hälfte der Ei-Sauce darauf verteilen, wieder zwei Blätter Teig.
        \step Danach wiederholend verkrümelten Schafskäse und Spinat, Teig, Sauce, Teig übereinander schichten.
        \step Die obersten Schichten müssen Teig, geriebener Käse, Teig sein. Zum Schluss Milch darauf streichen.
        \step Mit Sesam und Kümmel bestreuen und im vorgeheizten Backofen 25-30 Minuten bei {\unit[180]{\celsius}} backen.
    }
    \hint{%
        Alte Variante ist ohne Spinat, mit mehr Eiern, Dill und Petersilie.
    }
\end{recipe}
\end{document}
