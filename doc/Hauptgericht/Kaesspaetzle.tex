\documentclass[MeineRezepte.tex]{subfiles}
\begin{document}
\begin{recipe}[
    preparationtime = {\unit[25]{Min}},
    bakingtemperature = {\unit[75]{\celsius}},
    portion = {\portion{3}},
    source = Lili
]{Kässpätzle}\index{Kaesspaetzle@Kässpätzle}
    \graph{
        small=Kaesspaetzle_1,
        big=Kaesspaetzle_0
    }
    \ingredients{
        {\unit[200]{g}}    & Mehl\\
        2                  & Eier\\
        etwas              & Wasser, kalt\\
        {\unit[100]{g}}    & Gouda, gerieben\\
        3                  & Zwiebeln\\
                           & \\
                           &
    }
    \preparation{
        \step Aus dem Mehl, Eiern und Wasser einen zähflüssigen Teig rühren und ca. 15 Minuten stehen lassen.
        \step Zwiebeln klein würfeln und in Öl bei mittlerer Hitze rösten.
        \step Den Teig portionsweise mit der Spätzlepresse in kochendes Salzwasser drücken.
        \step Wenn die Spätzle oben schwimmen mit einer Schaumkelle in eine Schüssel geben, Käse darauf verteilen.
        \step Schritte 3 und 4 wiederholen bis der ganze Teig verarbeitet wurde; währenddessen im Ofen warm stellen.
        \step Zum Schluss die Zwiebeln darauf geben.
    }
    \hint{%
        Die (Glas-)Schüssel von Anfang an bei {\unit[75]{\celsius}} in den Ofen stellen. \newline
        Für Spinatspätzle kann man das halbe Gewicht wie Mehl an abgetropftem Spinat benutzen.
    }
\end{recipe}
\end{document}
