\documentclass[MeineRezepte.tex]{subfiles}
\begin{document}
\begin{recipe}[
    bakingtime = {\unit[30]{Min}},
    bakingtemperature = {\unit[200]{\celsius}},
    portion = {\portion{3}},
    source = Lili
]{Vollkornbuchteln mit Käsefüllung}\index{Vollkornbuchteln mit Kaesefüllung@Vollkornbuchteln mit Käsefüllung}
    \graph{
        big=VollkornBuchteln_0
    }
    \ingredients{
        Teig:                             & \\
        {\unit[300]{g}}                   & Vollkornmehl\\
        {\unit[1]{TL}}                    & Salz\\
        {\unit[$\nicefrac{1}{2}$]{TL}}    & Majoran\\
        1                                 & Ei\\
        {\unit[2]{EL}}                    & Öl\\
        {\unit[$\nicefrac{1}{8}$]{l}}     & Buttermilch\\
        etwas                             & Butter\\
                                          & Hefe\\
        Füllung:                          & \\
        {\unit[80]{g}}                    & Gouda\\
        {\unit[80]{g}}                    & Appenzeller\\% / Gruyère\\
        {\unit[80]{g}}                    & Kräuterfrischkäse\\
        {\unit[100]{g}}                   & Schlagsahne\\
        1                                 & Zwiebel\\
                                          & Schinken\\
                                          & Kresse
    }
    \preparation{
        \step Aus den Mehl und Gewürzen, Öl, Ei und Buttermilch und Hefe einen glatten Hefeteig kneten.
        \step Zudecken und an einen warmen Ort stellen; gehen lassen bis er doppelt so groß ist.
        \step Käse reiben und mit dem Frischkäse vermischen, Schinken und Zwiebeln würfeln und unter mischen.
        \step Den Teig nochmals durchkneten, in 6 Stücke teilen, in Fladen formen, füllen und zu Brötchen formen.
        \step Mit der Naht nach unten auf eine eingebutterte, ofenfeste Form legen.
        \step Sahne zugeben und 30~Minuten bei {\unit[200]{\celsius}} im vorgeheizten Ofen backen.
    }
    \hint{%
        Dazu passen Kräuter- oder Tomatensauce und grüner Salat.
    }
\end{recipe}
\end{document}
