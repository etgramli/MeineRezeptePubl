\makeatletter
\def\input@path{{../../}}
\makeatother
\providecommand{\main}{../../}
\documentclass[MeineRezepte.tex]{subfiles}
\begin{document}
\begin{recipe}[
    preparationtime = {\unit[15]{Min}},
    bakingtime = {\unit[100]{Min}},
    bakingtemperature = {\unit[180]{\celsius}},
    portion = {\portion{3}},
    source = Lili
]{Kartoffel-Reispfanne mit Speck}\index{Kartoffel-Reispfanne mit Speck}
    \graph{
        small=KartoffelReisPfanneSpeck_1,
        big=KartoffelReisPfanneSpeck_0
    }
    \ingredients{
        1                                      & Zwiebel\\
        {\unit[$1\nicefrac{1}{2}$]{Tassen}}    & Reis\\
        3                                      & Kartoffeln, groß\\
        {\unit[3]{Tassen}}                     & Wasser\\
        {\unit[4]{Streifen}}                   & Bauchspeck\\
        {\unit[3]{TL}}                         & Brühenpulver\\
                                               & Salz\\
                                               & Aromat\\
                                               &
    }
    \preparation{
        \step Die Zwiebel in Ringe schneiden und auf den Boden einer Auflaufform legen.
        \step Reis darüber verteilen und das Wasser mit der Brühe dazu gießen, bis er gerade bedeckt ist.
        \step Kartoffeln schälen, in Scheiben dünne schneiden und auf den Reis schichten. Dabei jede Schicht würzen.
        \step Den Bauchspeck bis zur Schwarte einschneiden und darauf legen.
        \step Alufolie über die Auflaufform spannen und 75 Minuten im Backofen backen.
        \step Die Alufolie entfernen und für die restliche Zeit backen, bis der Speck bräunt.
    }
    \hint{%
        Wenn der Speck knusprig werden soll den Grill einschalten, sobald die Alufolie weg ist.
    }
\end{recipe}
\end{document}
