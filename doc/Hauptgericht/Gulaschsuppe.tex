\documentclass[MeineRezepte.tex]{subfiles}
\begin{document}
\begin{recipe}[
    preparationtime = {\unit[30]{Min}},
    bakingtime = {\unit[90]{Min}},
    portion = {\portion{4}},
    source = Lili
]{Gulaschsuppe}\index{Gulaschsuppe}\index{Gulasch!Gulaschsuppe}
    \graph{
        big=Gulaschsuppe_0
    }
    \ingredients{
        {\unit[300]{g}}     & Rindfleisch\\
        2                   & Zwiebeln\\
        1                   & Paprika, rot\\
        {\unit[50]{g}}      & Sellerie\\
        {\unit[1]{EL}}      & Paprikapulver, scharf\\
        {\unit[500]{ml}}    & Wasser\\
        2                   & Kartoffeln\\
        3-4                 & Möhren\\
        {\unit[1]{P.}}      & Bratensauce / Brühe\\
        2                   & Lorbeerblätter\\
                            & Salz\\
                            & Pfeffer
    }
    \preparation{
        \step Fleisch in kleine Würfel schneiden und in heißem Öl kurz scharf anbraten, umfüllen.
        \step Zwiebeln, Paprika und Sellerie in Würfel schneiden und dünsten; Paprikapulver dazu und leicht anrösten.
        \step Mit Wasser ablöschen, Bratensauce einrühren und Fleisch wieder zugeben.
        \step Lorbeerblätter dazu geben und mindestens eine Stunde köcheln lassen.
        \step Kartoffeln und Möhren klein schneiden, dazu geben und mindestens $\nicefrac{1}{2}$ Stunde köcheln.
        \step Zum Schluss noch abschmecken.
    }
    \hint{%
        Für Gulasch nehme man {\unit[300]{g}} Fleisch pro Person\newline
        Man kann auch mit etwas Rotwein ablöschen (dann etwas weniger Wasser)
    }
\end{recipe}
\end{document}
