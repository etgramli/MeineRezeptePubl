\documentclass[MeineRezepte.tex]{subfiles}
\begin{document}
\begin{recipe}[
    preparationtime = {\unit[30]{h}},
    bakingtime = {\unit[2]{h}},
    bakingtemperature = {\unit[180]{\celsius}},
    portion = {\portion{6}},
    source = Oma Rita
]{Kaninchenbraten}\index{Kaninchenbraten}
    \graph{
        big=Kaninchenbraten_0
    }
    \ingredients{
        1                    & Kaninchen (Teile)\\
        1                    & Knoblauchzehe\\
        {\unit[1]{EL}}       & Rosmarin\\
        {\unit[1]{EL}}       & Thymian\\
        {\unit[1]{TL}}       & Pfeffer, schwarz, grob\\
        {\unit[4]{EL}}       & Öl\\
        {\unit[2]{EL}}       & Weinbrand (Cognac)\\
        $\nicefrac{1}{2}$    & Zitrone (nur Saft)\\
        1                    & Zwiebel\\
        {\unit[250]{ml}}     & Brühe
    }
    \preparation{
        \step Die Kaninchenteile in Knoblauch, Rosmarin, Thymian, Pfeffer, Öl, Weinbrand und Zitronensaft würzen.
        \step Die Teile einlegen und an einem kühlen Ort über Nacht lagern.
        \step Die Kaninchenteile salzen und pfeffern, in heißem Öl abraten und raus nehmen.
        \step Die Zwiebel im selben Öl dünsten und mit Brühe ablöschen.
        \step Kaninchen und Bratensauce in eine Schale tun und mit Deckel im Backofen bei {\unit[180]{\celsius}} schmoren.
    }
\end{recipe}
\end{document}
