\makeatletter
\def\input@path{{../../}}
\makeatother
\providecommand{\main}{../../}
\documentclass[MeineRezepte.tex]{subfiles}
\begin{document}
\begin{recipe}[
    preparationtime = {\unit[50]{Min}},
    bakingtime = {\unit[20]{Min}},
    portion = {\portion[Stück]{12}},
    source = Etienne
]{Schwäbische Maultaschen}\index{Schwaebische Maultaschen@Schwäbische Maultaschen}
    \graph{
        small=SchwaebischeMaultaschen_1,
        big=SchwaebischeMaultaschen_0
    }
    \ingredients{
        {\unit[300]{g}}      & Mehl\\
        2                    & Eier\\
        {\unit[60]{ml}}      & Wasser\\
        {\unit[1]{Prise}}    & Muskat\\
        {\unit[$\nicefrac{1}{2}$]{TL}}    & Salz\\
        {\unit[1]{EL}}       & Öl\\
        1                    & Zwiebel, groß\\
        1                    & Brötchen, altbacken\\
        {\unit[2]{TL}}       & Bärlauch-Pesto\\
        {\unit[125]{g}}      & Brät oder Hackfleisch\\
        {\unit[125]{g}}      & Blattspinat\\
        {\unit[1]{Prise}}    & Muskat\\
        1                    & Ei\\
        {\unit[1]{TL}}       & Salz\\
                             & Pfeffer
    }
    \preparation{
        \step Aus Mehl, Eiern, Wasser, Muskat, Salz und Öl einen geschmeidigen Nudelteig zubereiten.
        \step Zwiebel schälen, würfeln und glasig anschwitzen, währenddessen das Brötchen einweichen und ausdrücken.
        \step Dann alle Zutaten für die Füllung vermischen und mit Salz und gut mit Pfeffer abschmecken.
        \step Teig auf einer bemehlten Fläche auswallen und in Rechtecke schneiden.
        \step Füllung auf eine Hälfte geben und die andere Hälfte des Rechtecks darüber schlagen.
        \step Leicht andrücken, damit die Luft entweicht, dann fest andrücken, dass die Ränder aneinander haften.
        \step In viel Salzwasser etwa {\unit[20]{Minuten}} garen und dann abtropfen lassen.
    }
\end{recipe}
\end{document}
