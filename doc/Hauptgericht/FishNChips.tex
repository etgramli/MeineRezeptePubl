\makeatletter
\def\input@path{{../../}}
\makeatother
\providecommand{\main}{../../}
\documentclass[MeineRezepte.tex]{subfiles}
\begin{document}
\begin{recipe}[
    preparationtime = {\unit[40]{Min}},
    bakingtime = {\unit[20]{Min}},
    bakingtemperature = {\unit[160]{\celsius}},
    portion = {\portion{2}},
    source = Etienne
]{Fish'n'Chips}\index{Fish'n'Chips}\index{Pommes}
    \graph{
        big=FishNChips_0
    }
    \ingredients{
        4                   & Fischfilets (Kabeljau oder Schellfisch)\\
        5                   & Kartoffeln, groß\\
        {\unit[1]{l}}       & Öl zum Frittieren\\
        {\unit[150]{ml}}    & Bier\\
        {\unit[110]{g}}     & Mehl\\
        {\unit[1]{TL}}      & Backpulver\\
                            & Salz\\
                            & Pfeffer, schwarz\\
                            & Malzessig oder Zitronensaft
    }
    \preparation{
        \step Bier, Mehl, Backpulver und eine Priese Salz so lange verrühren bis keine Klümpchen mehr vorhanden sind.
        \step Die Marinade etwa 20 Minuten ziehen lassen. Solange die Kartoffeln schälen und in grobe Stifte schneiden.
        \step Die Stifte abtrocknen und bei mittlerer Hitze 4~Minuten frittieren. Darauf achten, dass sie nicht ankleben.
        \step Pommes rausnehmen, nochmals bei großer Hitze goldbraun frittieren und mit Salz und Malzessig würzen.
        \step Die Fischfilets trocken tupfen, mit Salz und Pfeffer würzen und mit Mehl bestreuen.
        \step Den Fisch durch die Marinade ziehen, langsam ins Öl legen und bei mittlerer Hitze goldbraun frittieren.
    }
    \hint{%
        Wenn man keinen Malzessig hat, kann man auch normalen Essig und Bier 1:1 mischen.\newline
        Pommes kann man auch aus Karotten machen.
    }
\end{recipe}
\end{document}
