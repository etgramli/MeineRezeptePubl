\makeatletter
\def\input@path{{../../}}
\makeatother
\providecommand{\main}{../../}
\documentclass[MeineRezepte.tex]{subfiles}
\begin{document}
\begin{recipe}[
    preparationtime = {\unit[45]{Min}},
    bakingtime = {\unit[25]{Min}},
    bakingtemperature = {\unit[200]{\celsius}},
    portion = {\portion{4}},
    source = Etienne
]{Rosenkohl-Schlemmer-Gratin}\index{Rosenkohl-Schlemmer-Gratin}
    \graph{
        small=RosenkohlSchlemmerGratin_1,
        big=RosenkohlSchlemmerGratin_0
    }
    \ingredients{
        Gratin:              & \\
        {\unit[1]{kg}}       & Rosenkohl\\
        6-8                  & Kartoffeln, festkoch.\\
        {\unit[150]{g}}      & Bergkäse\\
        {\unit[400]{ml}}     & Gemüsebrühe\\
                             & \\
        Sauce:               & \\
        {\unit[150]{ml}}     & Garflüssigkeit\\
        {\unit[100]{ml}}     & Weißwein, trocken\\
        {\unit[200]{ml}}     & Sahne\\
        {\unit[2]{EL}}       & Mehl\\
        {\unit[30]{g}}       & Butter\\
        {\unit[1]{Prise}}    & Muskat\\
                             & Salz\\
                             & Pfeffer
    }
    \preparation{
        \step Den Bergkäse in den Thermomix geben und 6~Sekunden auf Stufe~10 zerkleinern.
        \step Gemüsebrühe in den Topf geben und Garkorb mit {\unit[$\nicefrac{1}{2}$]{cm}}-dicken Kartoffeln-Scheiben einhängen.
        \step Dampfgaraufsatz mit dem geputztem Rosenkohl aufsetzen, 30-40~Minuten bei Stufe~1 auf Varoma garen.
        \step Kartoffeln und Rosenkohl in eine gefettete Auflaufform geben, vorsichtig vermischen. Brühe umfüllen.
        \step Die Butter 1~Minute bei {\unit[100]{\celsius}} andünsten, Mehl dazugeben und weitere 2~Minuten andünsten.
        \step 5~Minuten bei {\unit[100]{\celsius}}, Stufe~2 garen, nacheinander die Garflüssigkeit, Wein und Sahne dazu geben.
        \step Danach alles mit den Gewürzen abschmecken und 10~Sekunden auf Stufe~6 cremig rühren.
        \step Sauce über das Gemüse geben, mit Käse bestreuen und im Ofen bei {\unit[200]{\celsius}} 25~Minuten lang gratinieren.
    }
\end{recipe}
\end{document}
