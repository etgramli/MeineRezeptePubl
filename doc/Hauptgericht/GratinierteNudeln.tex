\makeatletter
\def\input@path{{../../}}
\makeatother
\providecommand{\main}{../../}
\documentclass[MeineRezepte.tex]{subfiles}
\begin{document}
\begin{recipe}[
    bakingtime = {\unit[20]{Min}},
    portion = {\portion{2}},
    source = Lili
]{Gratinierte Nudeln}\index{Gratinierte Nudeln}
    \graph{
        big=GratinierteNudeln_0
    }
    \ingredients{
        {\unit[200]{g}}      & Nudeln\\
        {\unit[250]{g}}      & Passierte Tomaten\\
        {\unit[100]{g}}      & Sahne\\
        etwas                & Schinken, gekocht\\
        $\nicefrac{1}{2}$    & Zwiebel\\
                             & Basilikum\\
                             & Oregano\\
                             & Salz\\
                             & Pfeffer\\
                             & Käse, gerieben\\
                             & Semmelbrösel
    }
    \preparation{
        \step Nudeln kochen; wenn sie fertig sind in eine Auflaufform geben.
        \step Zwiebel würfeln, würzen und in Öl köcheln; Sahne und klein gewürfelter Schinken dazu geben.
        \step Sauce mit den Nudeln mischen, ein Teil vom Käse untermischen, den Rest darauf verstreuen.
        \step Butterflöckchen und Semmelbrösel darauf geben und mit der Alu-Folie bedecken.
        \step Ca. 20 Minuten mit Grill (Programm 4) überbacken; nach {\unit[13]{Minuten}} die Alu-Folie entfernen.
    }
    \hint{%
        Weitere Verfeinerung: Statt Tomatensauce einfach 1 Dose Pizzatomaten unter die Nudeln msichen und kräftig mit italienischer Würzmischung, Oregano und Basilikum würzen
    }
\end{recipe}
\end{document}
