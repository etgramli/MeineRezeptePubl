\documentclass[MeineRezepte.tex]{subfiles}
\begin{document}
\begin{recipe}[
    preparationtime = {\unit[30]{Min}},
    bakingtime = {\unit[20]{Min}},
    bakingtemperature = {\unit[150]{\celsius}},
    portion = {\portion{4}},
    source = Afafe
]{Tajine}\index{Tajine}
    \graph{
        small=Tajine_1,
        big=Tajine_0
    }
    \ingredients{
        {\unit[400]{g}}                   & Rindergulasch\\
        1                                 & Zwiebel, rot\\
        2                                 & Knoblauchzehen\\
        {\unit[2]{TL}}                    & Kurkuma\\
        {\unit[1]{TL}}                    & Paprikapulver, süß\\
        {\unit[1]{TL}}                    & Paprikapulver, scharf\\
        {\unit[1]{Msp.}}                  & Kreuzkümmel, gemahlen\\
        {\unit[1]{Msp.}}                  & Ingwer, gemahlen\\
        {\unit[$\nicefrac{1}{2}$]{TL}}    & Zimt, gemahlen\\
        {\unit[800]{g}}                   & Tomaten, gehackt\\
                                          & Petersilie, frisch\\
                                          & Koriander, frisch\\
                                          & Salz\\
                                          & Pfeffer
    }
    \preparation{
        \step Die Zwiebel in Würfel schneiden und mit dem Knoblauch andünsten.
        \step Kurkuma und Paprikapulver dazu geben, leicht anrösten und mit den Tomaten ablöschen.
        \step Kreuzkümmel, Ingwer und Zimt dazu geben, vermischen und etwa 10~Minuten köcheln lassen.
        \step Die Sauce mit Salz und Pfeffer abschmecken und (ohne Deckel) zur Seite stellen.
        \step Das Gulasch in Salz, Pfeffer und Paprika wälzen, scharf anbraten und ohne Deckel abkühlen lassen.
        \step Das Fleisch in die Tajine geben, mit der Sauce bedecken und Petersilie und Koriander darauf streuen.
        \step Etwa 20~Minuten im Ofen bei mittlerer Hitze schmoren.
    }
    \hint{%
        Afafe macht dazu Kartoffeln, mit Reis schmeckt es auch gut.
    }
\end{recipe}
\end{document}
