\documentclass[MeineRezepte.tex]{subfiles}
\begin{document}
\begin{recipe}[
    preparationtime = {\unit[30]{Min}},
    bakingtime = {\unit[45]{Min}},
    bakingtemperature = {\unit[220]{\celsius}},
    portion = {\portion{4}},
    source = Lili
]{Moussaka}\index{Moussaka}
    \graph{
        small=Moussaka_1,
        big=Moussaka_0
    }
    \ingredients{
        {\unit[500]{g}}        & Rinderhackfleisch\\
        3                      & Auberginen\\
        1                      & Zwiebeln\\
        1                      & Knoblauchzehe\\
        3-4                    & Pellkartoffeln\\
        {\unit[1]{Dose}}       & Fleischtomaten\\
        {\unit[300-400]{g}}    & Schafskäse\\
        3                      & Eier\\
        {\unit[200]{g}}        & Creme Fraîche\\
        {\unit[2]{EL}}         & Tomatenmark\\
                               & Salz, Pfeffer\\
                               & Oregano\\
                               & Thymian\\
                               & Petersilie
    }
    \preparation{
        \step Auberginen waschen, in 1 cm dicke Längs-Scheiben schneiden und 12 runde quer-Scheiben schneiden.
        \step Zwiebel und Knoblauch klein schneiden und in Öl glasig dünsten; Hackfleisch dazu und durch braten.
        \step Tomatenwürfel dazu geben und so lange köcheln lassen bis Flüssigkeit etwas verdampft ist.
        \step Auberginen leicht in Olivenöl anbraten und abkühlen lassen.
        \step Hackfleischsauce mit Gewürzen abschmecken, Schafkäse dazu geben; Eier mit Creme Fraîche verquirlen.
        \step Folgendermaßen in eine Auflaufform schichten: Auberginen, $\nicefrac{1}{2}$ Hackfleischsauce, Pellkartoffelscheiben.
        \step Auberginen, $\nicefrac{1}{2}$ Hackfleischsauce, Eiersoße und die restlichen Auberginen.
        \step Im vorgeheizten Backofen bei {\unit[220]{\celsius}} 45 Minuten backen; 10 Minuten stehen lassen.
    }
\end{recipe}
\end{document}
