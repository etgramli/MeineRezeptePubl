\makeatletter
\def\input@path{{../../}}
\makeatother
\providecommand{\main}{../../}
\documentclass[MeineRezepte.tex]{subfiles}
\begin{document}
\begin{recipe}[
    preparationtime = {\unit[15]{Min}},
    bakingtime = {\unit[8]{Min}},
    portion = {\portion{3}},
    source = Etienne
]{Kartoffelpuffer}\index{Kartoffelpuffer}
    \graph{
        small=Kartoffelpuffer_1,
        big=Kartoffelpuffer_0
    }
    \ingredients{
        {\unit[1]{kg}}    & Kartoffeln\\
        2                 & Zwiebeln\\
        {\unit[1]{TL}}    & Johannisbrotkernmehl\\
                          & Öl\\
                          & Salz\\
                          & Pfeffer
    }
    \preparation{
        \step Zwiebeln in Würfel schneiden und andünsten. Währenddessen Kartoffeln schälen, waschen und raspeln.
        \step Kartoffeln und Zwiebeln mit dem Johannisbrotkernmehl vermischen und würzen.
        \step Masse portionsweise in eine Pfanne mit viel Öl geben, zu Fladen zerdrücken und goldbraun ausbacken.
        \step Noch warm mit Salz und Pfeffer servieren und eventuell nachwürzen.
    }
    \hint{%
        Etwas Würfelspeck passt gut dazu, auch direkt in der Masse.\newline
        Kann man kalt gut statt einer Brötchenhälfte für Lachsbrötchen benutzen.
    }
\end{recipe}
\end{document}
