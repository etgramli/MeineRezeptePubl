\documentclass[MeineRezepte.tex]{subfiles}
\begin{document}
\begin{recipe}[
    preparationtime = {\unit[90]{Min}},
    portion = {\portion{4}},
    source = Lili
]{Langosch}\index{Langosch}
    \graph{
        big=Langosch_0
    }
    \ingredients{
        Teig:                 & \\
        {\unit[350]{ml}}      & Wasser, lauwarm\\
        {\unit[1]{P.}}        & Trockenhefe\\
        {\unit[1]{Prise}}     & Zucker\\
        {\unit[500]{g}}       & Mehl\\
        {\unit[1]{TL}}        & Salz\\
                              & \\
        Dip:                  & \\
        {\unit[1]{Zehe}}      & Knoblauch\\
        {\unit[2]{Becher}}    & Joghurt\\
                              & Salz\\
                              & Pfeffer
    }
    \preparation{
        \step Hefe im Wasser und einer Prise Zucker auflösen und 15~Minuten gehen lassen.
        \step Mehl, Salz und Hefewasser in die Brotbackmaschine geben, kneten und 1~Stunde gehen lassen.
        \step Den Knoblauch pressen, mit dem Joghurt vermischen und mit Salz und etwas Pfeffer abschmecken.
        \step Kleine, dünne Fladen formen und in heißem Öl ausbacken.
        \step Noch warm mit der Knoblauchsauce servieren.
    }
    \hint{%
        Es wird sehr viel Teig daraus! - Man kann aus dem Rest noch ein Brot (mit Kümmel und Salz) machen.
    }
\end{recipe}
\end{document}
