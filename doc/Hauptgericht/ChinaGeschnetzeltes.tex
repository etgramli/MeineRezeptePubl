\makeatletter
\def\input@path{{../../}}
\makeatother
\providecommand{\main}{../../}
\documentclass[MeineRezepte.tex]{subfiles}
\begin{document}
\begin{recipe}[
    preparationtime = {\unit[20]{Min}},
    bakingtime = {\unit[15]{Min}},
    portion = {\portion{4}},
    source = Lili
]{China-Geschnetzeltes}\index{China-Geschnetzeltes}
    \graph{
        small=ChinaGeschnetzeltes_1,
        big=ChinaGeschnetzeltes_0
    }
    \ingredients{
        2                                 & Zwiebeln\\
        2                                 & Knoblauchzehen\\
        2-3                               & Gemüsepaprika\\
        1-2                               & Rüble\\
        1                                 & Kohlrabi\\
        3-4                               & Hähnchenbrustfilet\\
        {\unit[200]{g}}                   & Grüne Bohnen\\
                                          & Distelöl\\
                                          & Sesamöl\\
                                          & Sojasauce\\
        etwas                             & Wasser\\
        {\unit[1]{TL}}                    & Curry-Pulver\\
        {\unit[1]{Msp.}}                  & Ingwer-Pulver\\
        {\unit[$\nicefrac{1}{2}$]{TL}}    & Sambal Olek
    }
    \preparation{
        \step Öl in der Pfanne erhitzen, geschnetzeltes Gemüse dazu geben und bei Stufe 2 schmoren lassen.
        \step Immer wieder umrühren. Wenn das Gemüse etwas weich wird, ein bisschen Wasser dazu geben.
        \step Fleisch in Würfel schneiden und mit Soja-Sauce beträufeln und ein paar Minuten stehen lassen.
        \step Dann zum Gemüse geben und mitbraten; mit Soja-Sauce und den Gewürzen abschmecken.
    }
    \hint{%
        Basmati-Reis schmeckt sehr gut dazu.
    }
\end{recipe}
\end{document}
