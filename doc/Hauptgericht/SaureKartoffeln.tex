\makeatletter
\def\input@path{{../../}}
\makeatother
\providecommand{\main}{../../}
\documentclass[MeineRezepte.tex]{subfiles}
\begin{document}
\begin{recipe}[
    bakingtime={\unit[30]{Min}},
    portion={\portion{4}},
    source=Oma Tilda
]{Saure Kartoffeln}\index{Saure Kartoffeln}
    \graph{
        big=SaureKartoffeln_0
    }
    \ingredients{
        {\unit[500]{g}}     & Kartoffeln\\
                            & Brühe\\
        1                   & Lorbeerblatt\\
                            & Mehl\\
        1                   & Knoblauchzehe\\
        {\unit[1]{TL}}      & Paprikapulver\\
        {\unit[1]{Glas}}    & Wasser\\
        {\unit[2]{EL}}      & Essig\\
                            & Öl\\
                            & Salz
    }
    \preparation{
        \step Die Kartoffeln in Scheiben schneiden und in einem Topf mit Wasser bedecken.
        \step Brühe und Lorbeer dazu geben und 30 Minuten köcheln lassen, bis die Kartoffeln weich sind.
        \step In der Zwischenzeit das Mehl in dem Öl leicht anschwitzen.
        \step Die Knoblauchzehe hinein pressen, das Paprikapulver dazu geben und auch leicht anschwitzen.
        \step Wenn das Mehl anfängt leicht zu riechen, mit einem Glas Wasser ablöschen.
        \step Mit dem Essig und etwas Salz abschmecken; dann die Sauce mit den Kartoffeln mischen.
    }
\end{recipe}
\end{document}
