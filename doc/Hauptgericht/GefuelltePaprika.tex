\documentclass[MeineRezepte.tex]{subfiles}
\begin{document}
\begin{recipe}[
    preparationtime = {\unit[20]{Min}},
    bakingtime = {\unit[40]{Min}},
    portion = {\portion{4}},
    source = Oma Tilda
]{Gefüllte Paprika}\index{Gefuellte Paprika@Gefüllte Paprika}
    \graph{
        big=GefuelltePaprikaHack_0
    }
    \ingredients{
        4                      & Paprika, rot / gelb\\
        {\unit[300]{g}}        & Hackfleisch\\
        etwas                  & Brot, alt\\
        {\unit[1]{Portion}}    & Reis, gekocht\\
        {\unit[250]{ml}}       & Brühe\\
        {\unit[1]{Becher}}     & Saure Sahne\\
        1                      & Zwiebel\\
        1                      & Knoblauchzehe\\
        1                      & Ei\\
                               & Salz\\
                               & Pfeffer\\
                               & Thymian
    }
    \preparation{
        \step Brot einweichen und ausdrücken; Zwiebel und Knoblauch klein hacken.
        \step Aus Hackfleisch, Reis, Brot, Zwiebel, Knoblauch und Gewürzen einen homogenen Fleischteig zubereiten.
        \step Deckel der Paprika abschneiden, die Kerne entfernen und füllen. Den Deckel wieder aufsetzen.
        \step Brühe zum kochen bringen, Sahne einrühren, Paprika reinsetzen und 40 Minuten kochen.
    }
    \hint{%
        Reste vom Hackfleischteig werden zu Hackfleischküchlein.
    }
\end{recipe}
\end{document}
