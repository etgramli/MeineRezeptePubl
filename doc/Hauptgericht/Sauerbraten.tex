\makeatletter
\def\input@path{{../../}}
\makeatother
\providecommand{\main}{../../}
\documentclass[MeineRezepte.tex]{subfiles}
\begin{document}
\begin{recipe}[
    preparationtime = {\unit[30]{Min}},
    bakingtime = {\unit[3]{h}},
    portion = {\portion{4}},
    source = Etienne,
]{Sauerbraten}\index{Sauerbraten}
    \graph{
        big=Sauerbraten_0
    }
    \ingredients{
        {\unit[1]{kg}}      & Rindfleisch\\
        1                   & Zwiebel\\
        2                   & Knoblauchzehen\\
        2                   & Nelken\\
        2                   & Lorbeerblätter\\
        10                  & Pfefferkörner\\
        3                   & Wacholderbeeren\\
        {\unit[300]{ml}}    & Rotwein\\
        {\unit[150]{ml}}    & Rotweinessig\\
                            & \\
        2                   & Zwiebeln\\
        {\unit[1]{EL}}      & Tomatenmark\\
        {\unit[1]{TL}}      & Brühenpulver\\
    }
    \preparation{
        \step Gemüse klein schneiden und mit Fleisch, Gewürzen, Wein und Essig 2-3 Tage im Kühlschrank einlegen.
        \step Die Marinade abgießen, die Gewürze wegwerfen und die Flüssigkeit behalten.
        \step Fleisch anbraten und die zwei Zwiebeln würfeln und ebenfalls anbraten, das Tomatenmark kurz anrösten.
        \step Mit der Marinade ablöschen, Brühenpulver dazu geben und 2,5~Stunden schmoren.
    }
\end{recipe}
\end{document}
