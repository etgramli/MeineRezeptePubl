\makeatletter
\def\input@path{{../../}}
\makeatother
\providecommand{\main}{../../}
\documentclass[MeineRezepte.tex]{subfiles}
\begin{document}
\begin{recipe}[
    preparationtime = {\unit[15]{Min}},
    bakingtime = {\unit[60]{Min}},
    portion = {\portion{4}},
    source = Etienne
]{Lammbraten an Sherry-Sauce}\index{Lammbraten an Sherry-Sauce}
    \graph{
        small=LammbratenSherrySauce_1,
        big=LammbratenSherrySauce_0
    }
    \ingredients{
        {\unit[1]{kg}}      & Lammkeule\\
        1                   & Zwiebel, groß\\
        {\unit[250]{ml}}    & Sherry\\
        {\unit[2]{EL}}      & Tomatenmark\\
        {\unit[1]{Bund}}    & Rosmarin\\
        {\unit[1]{Bund}}    & Thymian\\
        {\unit[200]{ml}}    & Sahne\\
                            & Salz\\
                            & Pfeffer
    }
    \preparation{
        \step Das Lamm salzen, pfeffern und in heißem Fett von allen Seiten anbraten.
        \step Die Zwiebel in Würfel schneiden und leicht andünsten, Tomatenmark dazu geben und kurz rösten.
        \step Mit dem Sherry ablöschen. Rosmarin, Thymian und Sahne dazu geben und eine Stunde köcheln lassen.
        \step Ab und zu Sauce über das Fleisch träufeln, damit es nicht austrocknet.
        \step Das Fleisch herausnehmen und in Scheiben schneiden, die Sauce mit Salz und Pfeffer abschmecken.
        \step Das Fleisch wieder hinein legen, kurz ziehen lassen und dann servieren.
    }
    \hint{%
        Dazu passen gut Reis oder Bratkartoffeln und grüne Bohnen.
    }
\end{recipe}
\end{document}
