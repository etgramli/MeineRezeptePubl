\makeatletter
\def\input@path{{../../}}
\makeatother
\providecommand{\main}{../../}
\documentclass[MeineRezepte.tex]{subfiles}
\begin{document}
\begin{recipe}[
    preparationtime = {\unit[90]{Min}},
    bakingtime = {\unit[40]{Min}},
    bakingtemperature = {\unit[180]{\celsius}},
    portion = {\portion{4}},
    source = Lili
]{Hirse-Oliven-Auflauf}\index{Hirse-Oliven-Auflauf}
    \graph{
        small=HirseOlivenAuflauf_1,
        big=HirseOlivenAuflauf_0
    }
    \ingredients{
        {\unit[200]{g}}        & Hirse\\
        {\unit[700]{ml}}       & Brühe\\
        1                      & Knoblauchzehe\\
        {\unit[2]{Stangen}}    & Lauch\\
        1                      & Paprika, rot\\
        {\unit[2]{EL}}         & Olivenöl\\
        20                     & Oliven\\
        3                      & Eier\\
        {\unit[200]{ml}}       & Sahne und Milch\\
        {\unit[50]{g}}         & Käse, gerieben\\
                               & Muskatnuss\\
                               & Pfeffer\\
                               & Salz\\
                               & Oregano\\
                               & Thymian
    }
    \preparation{
        \step Hirse in Gemüsebrühe 6~Minuten kochen lassen und 10~Minuten lang garen; Backofen vorheizen.
        \step Paprika und Lauch putzen und in {\unit[1]{cm}} breite Ringe respektive Streifen schneiden.
        \step Knoblauch schälen und würfeln; mit Lauch und Paprika in Olivenöl anbraten; Kräuter hinzufügen.
        \step Oliven halbieren und mit dem Gemüse und Hirse mischen und in eine eingefettete Auflaufform geben.
        \step Milch, Sahne und Gewürze verrühren, auf der Hirse verteilen und mit dem Käse bestreuen.
        \step 35-40~Minuten backen.
    }
\end{recipe}
\end{document}
