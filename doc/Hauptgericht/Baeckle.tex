\makeatletter
\def\input@path{{../../}}
\makeatother
\providecommand{\main}{../../}
\documentclass[MeineRezepte.tex]{subfiles}
\begin{document}
\begin{recipe}[
    preparationtime = {\unit[30]{Min}},
    bakingtime = {\unit[3]{h}},
    bakingtemperature = {\unit[160]{\celsius}},
    portion = {\portion{2}},
    source = Lili
]{Bäckle}\index{Baeckle@Bäckle}
    \graph{
        small=Baeckle_1,
        big=Baeckle_0
    }
    \ingredients{
        4                   & Bäckle vom Kalb\\
        1-2                 & Zwiebeln\\
        1                   & Knoblauchzehe\\
        1                   & Paprika\\
        2                   & Rüble\\
        handvoll            & Sellerie\\
        {\unit[2]{EL}}      & Tomatenmark\\
        {\unit[300]{ml}}    & Rotwein, trocken\\
        {\unit[300]{ml}}    & Brühe\\
        1                   & Lorbeerblatt\\
                            & Creme Fraîche\\
                            & Salz\\
                            & Pfeffer
    }
    \preparation{
        \step Die Bäckle scharf in Öl anbraten, rausnehmen und zur Seite stellen.
        \step Das Gemüse in Würfel schneiden und in derselben Pfanne andünsten, bis es leicht bräunt.
        \step Tomatenmark dazu geben, kurz anrösten und mit Wein und Brühe ablöschen.
        \step Die Bäckle zum Gemüse in die Sauce; sie sollten fast mit Flüssigkeit bedeckt sein.
        \step Lorbeer dazu geben und etwa {\unit[3]{Stunden}} bei {\unit[160]{\celsius}} schmoren, aber nicht kochen.
        \step Bäckle vorsichtig herausnehmen, die Sauce pürieren und Creme Fraîche dazu rühren.
        \step Mit Salz und Pfeffer abschmecken und mit den Bäckle servieren.
    }
\end{recipe}
\end{document}
