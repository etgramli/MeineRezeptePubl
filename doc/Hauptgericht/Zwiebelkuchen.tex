\makeatletter
\def\input@path{{../../}}
\makeatother
\providecommand{\main}{../../}
\documentclass[MeineRezepte.tex]{subfiles}
\begin{document}
\begin{recipe}[
    bakingtime = {\unit[30]{Min}},
    bakingtemperature = {\unit[220]{\celsius}},
    source = Lili
]{Zwiebelkuchen}\index{Zwiebelkuchen}
    \graph{
        small=Zwiebelkuchen_1,
        big=Zwiebelkuchen_0
    }
    \ingredients{
        {\unit[300]{g}}     & Mehl\\
        {\unit[150]{ml}}    & Milch, lauwarm\\
        {\unit[1]{TL}}      & Salz\\
        {\unit[80]{g}}      & Butter, warm\\
        {\unit[20]{g}}      & Hefe\\
        {\unit[600]{g}}     & Zwiebeln\\
        {\unit[100]{g}}     & Schinken\\
        2                   & Eier\\
        {\unit[250]{ml}}    & Milch-Sahne-Mix\\
        {\unit[100]{g}}     & Saure Sahne \\
                            & Käse, gerieben\\
                            & Paprikapulver\\
                            & Kümmel \\
                            & Salz, Pfeffer
    }
    \preparation{
        \step Aus Mehl, Milch, Salz, Butter, Hefe einen Hefeteig zubereiten und mindestens 30 Minuten gehen lassen.
        \step Zwiebeln und Schinken würfeln, in etwas Butter dünsten, bis Zwiebeln halb-weich sind.
        \step Eier verquirlen, Milch-Sahne-Gemisch, Sahne und Käse dazu geben.
        \step Zwiebeln und Schinken unterrühren und dann salzen und pfeffern.
        \step Teig auf ein Backblech ausrollen, nochmals etwas gehen lassen, Zwiebelmasse darauf.
        \step Im vorgeheizten Backofen 30 Minuten bei {\unit[220]{\celsius}} backen.
    }
\end{recipe}
\end{document}
