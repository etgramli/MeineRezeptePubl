\documentclass[MeineRezepte.tex]{subfiles}
\begin{document}
\begin{recipe}[
    preparationtime={\unit[20]{Min}},
    portion = {\portion{4}},
    source = Oma Tilda
]{Granatiermasch}\index{Granatiermasch}
    \graph{
        big=Granatiermasch_0
    }
    \ingredients{
        1                       & Zwiebel\\
        {\unit[1]{TL}}          & Paprikapulver\\
        {\unit[500]{ml}}        & Brühe\\
        6                       & Kartoffeln\\
        1                       & Lorbeerblatt\\
        {\unit[150]{g}}         & Nudeln\\
        {\unit[4]{Scheiben}}    & Brot (alt)\\
                                & Salz\\
                                &
    }
    \preparation{
        \step Die Zwiebel andünsten, das darin Paprikapulver kurz anrösten und mit der Brühe ablöschen.
        \step Die Kartoffeln in Würfel schneiden, in die Brühe geben und mit dem Lorbeerblatt weich kochen.
        \step Inzwischen die Nudeln kochen, abgießen aber etwas Kochwasser aufheben.
        \step Die Kartoffeln zerstampfen, aber nicht zu Mus. Es können noch Stückchen da bleiben.
        \step Die Nudeln dazu geben und wenn es zu trocken ist noch etwas Kochwasser dazu tun.
        \step Das Brot in Öl anrösten, salzen und auf das Essen legen.
    }
    \hint{%
        Gut um Brotreste zu verwerten.
    }
\end{recipe}
\end{document}
