\makeatletter
\def\input@path{{../../}}
\makeatother
\providecommand{\main}{../../}
\documentclass[MeineRezepte.tex]{subfiles}
\begin{document}
\begin{recipe}[
    bakingtime = {\unit[45]{Min}},
    bakingtemperature = {\unit[200]{\celsius}},
    portion = {\portion{4}},
    source = Lili
]{Quiche aus Quark-Öl-Teig}\index{Quiche aus Quark-Oel-Teig@Quiche aus Quark-Öl-Teig}
    \graph{
        small=QuicheAusQuarkOElTeig_1,
        big=QuicheAusQuarkOElTeig_0
    }
    \ingredients{
        Teig:                             & \\
        {\unit[200]{g}}                   & Mehl\\
        {\unit[$\nicefrac{1}{2}$]{P.}}    & Backpulver\\
        {\unit[$\nicefrac{1}{2}$]{TL}}    & Salz\\
        {\unit[100]{g}}                   & Quark\\
        {\unit[3]{EL}}                    & Öl\\
        1                                 & Ei\\
                                          & \\
        Belag:                            & \\
                                          & Schafkäse\\
                                          & Spinat\\
                                          & Kirschtomaten\\
                                          & \\
        Guss:                             & \\
        {\unit[200]{g}}                   & Sahne\\
        3                                 & Eier\\
                                          & Salz\\
                                          & Pfeffer
    }
    \preparation{
        \step Die Zutaten für den Teig in den Mixtopf geben und 1 Minute bei Teigstufe vermischen.
        \step Den Teig in einer Form verteilen und den Belag darauf verteilen.
        \step Die Zutaten für den Guss vermischen und auf dem Rest verteilen.
        \step Die Quiche 45 Minuten bei {\unit[200]{\celsius}} backen.
    }
    \hint{%
        Statt Sahne geht auch Schmand oder Mascarpone.
    }
\end{recipe}
\end{document}
