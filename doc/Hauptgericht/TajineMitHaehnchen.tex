\documentclass[MeineRezepte.tex]{subfiles}
\begin{document}
\begin{recipe}[
    preparationtime = {\unit[20]{Min}},
    bakingtime = {\unit[40]{Min}},
    bakingtemperature = {\unit[150]{\celsius}},
    portion = {\portion{4}},
    source = Afafe
]{Tajine mit Hähnchen}\index{Tajine mit Hähnchen}
    \graph{
        small=TajineMitHaehnchen_1,
        big=TajineMitHaehnchen_0
    }
    \ingredients{
        2                    & Hähnchenbrust\\%350g
        2                    & Zwiebeln, mittelgroß\\
        2                    & Knoblauchzehen\\
        $\nicefrac{1}{2}$    & Paprika, rot\\
        {\unit[2]{TL}}       & Kurkuma\\
        {\unit[1]{TL}}       & Paprikapulver, süß\\
        {\unit[1]{Msp.}}     & Kreuzkümmel, gemahlen\\
        {\unit[1]{Msp.}}     & Ingwer, gemahlen\\
        1                    & Kartoffel, groß\\
        1                    & Rübe, klein\\
        1                    & Tomate\\
        $\nicefrac{1}{4}$    & Salzzitrone\\
        {\unit[100]{g}}      & Erbsen\\
        etwas                & Wasser\\
                             & Petersilie, frisch\\
                             & Koriander, frisch\\
                             & Olivenöl\\
                             & Salz\\
                             & Pfeffer
    }
    \preparation{
        \step Zwiebel in halbe Ringe und Paprika in Streifen schneiden und den Knoblauch in klein hacken.
        \step Alles drei in Olivenöl andünsten, die Hühnerbrust in Streifen schneiden und kurz anbraten.
        \step Die Gewürze dazu geben und ebenfalls kurz anrösten, dann alles in die Tajine legen.
        \step Kartoffel in dünne Scheiben hobeln und an den Rand der Tajine legen.
        \step Die Rübe ebenfalls in dünne Scheiben hobeln und darauf verteilen.
        \step Etwas Wasser in die Tajine geben sodass der Boden bedeckt ist.
        \step Die Tomate in Scheiben und die Salzzitrone in Schnitze schneiden und mit den Erbsen auf den Rest legen.
        \step Mit Petersilie und Koriander würzen, mit Frischhaltefolie bedecken und 40~Minuten im Ofen garen.
    }
\end{recipe}
\end{document}
