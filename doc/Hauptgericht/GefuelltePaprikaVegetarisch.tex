\makeatletter
\def\input@path{{../../}}
\makeatother
\providecommand{\main}{../../}
\documentclass[MeineRezepte.tex]{subfiles}
\begin{document}
\begin{recipe}[
    preparationtime = {\unit[20]{Min}},
    bakingtime = {\unit[45]{Min}},
    bakingtemperature = {\unit[200]{\celsius}},
    portion = {\portion{4}},
    source = Lili
]{Gefüllte Paprika (vegetarisch)}\index{Gefuellte Paprika@Gefüllte Paprika!Gefuellte Paprika vegetarisch@Gefüllte Paprika (vegetarisch)}
    \graph{
        small=GefuelltePaprika_1,
        big=GefuelltePaprika_0
    }
    \ingredients{
        6                  & Paprika, rot\\
        {\unit[200]{g}}    & Schafskäse\\
        {\unit[500]{g}}    & Passierte Tomaten\\
        1                  & Zwiebel\\
        1                  & Knoblauchzehe\\
        1                  & Ei\\
                           & Olivenöl\\
                           & Salz\\
                           & Pfeffer\\
                           & Petersilie
    }
    \preparation{
        \step Die Deckel der Paprika entfernen und die Samen entfernen.
        \step Den Knoblauch und die Zwiebel klein schneiden und andünsten.
        \step Schafskäse würfeln, mit Zwiebel, Knoblauch und Ei vermischen und würzen.
        \step Paprikaschoten damit füllen und den Deckel leicht darauf drücken.
        \step Die Paprika mit den Tomaten in eine Kasserolle legen und ca. 45 Minuten bei {\unit[200]{\celsius}} im Ofen backen.
    }
\end{recipe}
\end{document}
