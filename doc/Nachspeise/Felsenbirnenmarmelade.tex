\documentclass[MeineRezepte.tex]{subfiles}
\begin{document}
\begin{recipe}[
    preparationtime = {\unit[60]{Min}},
    bakingtime = {\unit[20]{Min}},
    portion = {\portion[Gläser]{2}},
    source = Etienne
]{Felsenbirnenmarmelade}\index{Felsenbirnenmarmelade}
    \graph{
        small=Felsenbirnenmarmelade_1,
        big=Felsenbirnenmarmelade_0
    }
    \ingredients{
        {\unit[550]{g}}     & Felsenbirnen\\
        {\unit[50]{g}}      & Rhabarber\\
        etwas               & Wasser\\
        {\unit[1]{Msp.}}    & Vanille, gemahlen\\
        {\unit[200]{g}}     & Gelierzucker (3:1)
    }
    \preparation{
        \step Felsenbirnen und Rhabarber waschen, in etw. Wasser weich kochen und durch einen groben Sieb passieren.
        \step Vanille und {\unit[4]{EL}} vom Trester dazu geben, mit dem Gelierzucker mischen und unter Rühren aufkochen.
        \step In Gläser abfüllen und auf dem Kopf stehend abkühlen lassen.
    }
    \hint{%
        Mit dem Trester bekommt die Marmelade den Marzipan-Geschmack der Kerne.
    }
\end{recipe}
\end{document}
