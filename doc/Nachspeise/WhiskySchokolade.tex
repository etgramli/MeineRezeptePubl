\documentclass[MeineRezepte.tex]{subfiles}
\begin{document}
\begin{recipe}[
    portion = {\portion[Tafel]{1}},
    source = Lili
]{Whisky-Schokolade}\index{Whisky-Schokolade}
    \graph{
        small=WhiskySchokolade_1,
        big=WhiskySchokolade_0
    }
    \ingredients{
        {\unit[60]{g}}      & Kakaobutter\\
        {\unit[2-3]{TL}}    & Kakaopulver\\
        {\unit[30]{g}}      & Rosinen\\
        {\unit[20]{ml}}     & Whisky, sherry"-fass"-gereift\\
        {\unit[1]{Msp.}}    & Zimt\\
        {\unit[1]{Msp.}}    & Nelke\\
        {\unit[1]{Msp.}}    & Vanille\\
        etwas               & Zucker
    }
    \preparation{
        \step Rosinen über Nacht in einem sherryfassgereiften Whisky einlegen.
        \step Kakaobutter im Wasserbad schmelzen, das Kakaopulver darin auflösen und mit Gewürzen abschmecken.
        \step Rosinen in einer Form auslegen und die Schokolade darauf verteilen.
        \step Form schließen und im Kühlschrank kalt stellen bis die Schokolade hart ist.
    }
    \hint{%
        Beim Abkühlen setzt sich die Kakaobutter etwas ab und bildet eine weiße Schicht.
    }
\end{recipe}
\end{document}
