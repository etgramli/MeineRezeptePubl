\documentclass[MeineRezepte.tex]{subfiles}
\begin{document}
\begin{recipe}[
    bakingtime = {\unit[30]{Min}},
    bakingtemperature = {\unit[200]{\celsius}},
    portion = {\portion[Blech]{1}},
    source = Oma Tilda
]{Donauwelle}\index{Donauwelle}
    \graph{
        small=DonauwelleTilda_1,
        big=DonauwelleTilda_0
    }
    \ingredients{
        {\unit[250]{g}}                    & Butter\\
        {\unit[250]{g}}                    & Zucker\\
        5                                  & Eier\\
        {\unit[2]{P.}}                     & Vanille-Zucker\\
        {\unit[380]{g}}                    & Mehl\\
        etwas                              & Milch\\
        {\unit[$\nicefrac{1}{2}$]{P.}}     & Backpulver\\
        {\unit[$1\nicefrac{1}{2}$]{EL}}    & Kakao\\
        {\unit[2]{Gl.}}                    & Sauerkirschen\\
        {\unit[1]{P.}}                     & Vanille-Pudding\\
        {\unit[40]{g}}                     & Zucker\\
        {\unit[600]{ml}}                   & Milch\\
        {\unit[250]{g}}                    & Voll"-milch"-scho"-ko"-la"-de\\
        etwas                              & Butter
    }
    \preparation{
        \step Zucker, Butter, Eier und Vanille-Zucker schaumig rühren.
        \step Mehl und Backpulver nach und nach unterrühren, bei Bedarf Milch dazugeben.
        \step Die eine Hälfte des Teigs auf ein gefettetes Blech schmieren.
        \step Die andere Hälfte mit Kakao mischen und auf den hellen Teig geben.
        \step Die abgetropften Sauerkirschen dazu und bei {\unit[200]{\celsius}} ca. 30 Minuten backen, abkühlen lassen.
        \step Abgekühlten Pudding darauf verteilen und dann mit geschmolzener Schokolade glasieren.
    }
\end{recipe}
\end{document}
