\makeatletter
\def\input@path{{../../}}
\makeatother
\providecommand{\main}{../../}
\documentclass[MeineRezepte.tex]{subfiles}
\begin{document}
\begin{recipe}[
    bakingtime = {\unit[30]{Min}},
    bakingtemperature = {\unit[160]{\celsius}},
    source = Lili
]{Whisky-Schokoladen-Kuchen (ketogen)}\index{Whisky-Schokoladen-Kuchen}
    \graph{
        big=WhiskySchokoladenKuchenKetogen_0
    }
    \ingredients{
        2                                 & Eier\\
        {\unit[80]{g}}                    & Zucker\\
        {\unit[125]{g}}                   & Butter\\
        {\unit[1]{Msp.}}                  & Bourbon-Vanille\\
        {\unit[1]{TL}}                    & Zimt\\
        {\unit[1]{TL}}                    & Kakao\\
        {\unit[1]{Msp.}}                  & Nelke\\
        {\unit[1]{Msp.}}                  & Kardamon\\
        {\unit[1]{Msp.}}                  & Ingwer\\
        {\unit[65]{g}}                    & Mandelmehl\\
        {\unit[1]{TL}}                    & Johannisbrotkernmehl\\
        {\unit[$\nicefrac{1}{2}$]{P.}}    & Backpulver\\
        {\unit[60]{g}}                    & Rosinen\\
        {\unit[60]{ml}}                   & Whisky\\
        {\unit[50]{g}}                    & Schokolade\\
        {\unit[1]{TL}}                    & Kokosöl\\
        {\unit[3]{TL}}                    & Xucker\\
    }
    \preparation{
        \step Die Rosinen im Whisky einlegen, bis sie vollgesogen sind.
        \step Butter, Zucker, Vanille-Zucker und Eier schaumig rühren.
        \step Die trockenen Zutaten vermischen, unterrühren und in eine gut gefettete 17-cm-Springform füllen.
        \step Bei {\unit[160]{\celsius}} im vorgeheizten Backofen ca. 30 Minuten backen.
        \step Schokolade mit Kokosöl und Xucker bei ca. {\unit[50]{\celsius}} schmelzen und den abgekühlten Kuchen glasieren.
    }
\end{recipe}
\end{document}
