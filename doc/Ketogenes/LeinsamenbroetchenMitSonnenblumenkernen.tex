\makeatletter
\def\input@path{{../../}}
\makeatother
\providecommand{\main}{../../}
\documentclass[MeineRezepte.tex]{subfiles}
\begin{document}
\begin{recipe}[
    preparationtime = {\unit[15]{Min}},
    bakingtime = {\unit[35]{Min}},
    bakingtemperature = {\unit[180]{\celsius}},
    portion = {\portion[Stück]{9}},
    source = Lili
]{Leinsamenbrötchen (Sonnenblumenkerne)}\index{Leinsamenbrötchen mit Sonnenblumenkernen}\index{Brot!Leinsamenbrötchen mit Sonnenblumenkernen}
    \graph{
        big=LeinsamenbroetchenMitSonnenblumenkernen_0
    }
    \ingredients{
        {\unit[250]{g}}     & Quark\\
        {\unit[150]{ml}}    & Wasser\\
        {\unit[100]{g}}     & Leinsamenmehl\\
        {\unit[100]{g}}     & Sonnenblumenkerne\\
        {\unit[40]{g}}      & Leinsamen (geschrotet)\\
        {\unit[30]{g}}      & Flohsamenschalen\\
        3                   & Eier\\
        {\unit[2]{TL}}      & Backpulver\\
        {\unit[1]{TL}}      & Salz\\
        {\unit[1]{TL}}      & Johannisbrotkernmehl
    }
    \preparation{
        \step Alle Zutaten vermischen und den Teig 10~Minuten ruhen lassen.
        \step Im vorgeheizten Backofen {\unit[35]{Minuten}} bei {\unit[180]{\celsius}} Umluft backen und im Ofen abkühlen lassen.
    }
    \hint{%
        Aus {\unit[600]{ml}} Wasser und {\unit[4]{EL}} Natron kann man Lauge für Laugenbrötchen machen. Dazu muss man sie kurz aufkochen.
    }
\end{recipe}
\end{document}
