\documentclass[MeineRezepte.tex]{subfiles}
\begin{document}
\begin{recipe}[
    bakingtime = {\unit[70-80]{Min}},
    bakingtemperature = {\unit[150]{\celsius}},
    portion = {\portion[Blech]{1}},
    source = Lili
]{Keto-Knäckebrot}\index{Knaeckebrot@Knäckebrot}\index{Brot!Knaeckebrot@Knäckebrot}
    \graph{
        big=KetoKnaeckebrot_0
    }
    \ingredients{
        {\unit[50]{g}}                    & Son"-nen"-blu"-men"-kerne\\
        {\unit[50]{g}}                    & Kürbiskerne\\
        {\unit[40]{g}}                    & Leinsamen, gemahlen\\
        {\unit[20]{g}}                    & Flohsamenschalen\\
        {\unit[50]{g}}                    & (Schwarz-) Kümmel, Mohn oder Sesam\\
        {\unit[10]{g}}                    & Kleie\\
        {\unit[$\nicefrac{1}{2}$]{TL}}    & Meersalz\\
        {\unit[30]{g}}                    & Olivenöl\\
        {\unit[300]{ml}}                  & Wasser\\
        opt.                              & Parmesan\\
        opt.                              & Röstzwiebeln\\
        opt.                              & Kurkuma\\
        opt.                              & Pfeffer, schwarz
    }
    \preparation{
        \step Alle Zutaten vermischen und auf ein gut gebuttertes Backblech gleichmäßig streichen.
        \step Im vorgeheizten Backofen {\unit[35-40]{Min}} bei {\unit[150]{\celsius}} vor-backen.
        \step Danach in gewünschte Größe zerschneiden und nochmal {\unit[35-40]{Min}} backen.
    }
    \hint{%
        Ein Blech entspricht ca. {\unit[280]{g}} Knäckebrot; {\unit[100]{g}} Knäckebrot entsprechen {\unit[6]{g}} Kohlenhydraten.
    }
\end{recipe}
\end{document}
