\makeatletter
\def\input@path{{../../}}
\makeatother
\providecommand{\main}{../../}
\documentclass[MeineRezepte.tex]{subfiles}
\begin{document}
\begin{recipe}[
    preparationtime = {\unit[10]{Min}},
    bakingtime = {\unit[50]{Min}},
    bakingtemperature = {\unit[160]{\celsius}},
    portion = {\portion[Stollen]{1}},
    source = Lili
]{Stollen}\index{Stollen}
    \graph{
        big=Stollen_0
    }
    \ingredients{
        {\unit[125]{g}}                       & Quark\\
        {\unit[80]{g}}                        & Xylith\\
        {\unit[80]{g}}                        & Butter\\
        2                                     & Eier\\
        {\unit[250]{g}}                       & Mandeln, gemahlen\\
        {\unit[1]{TL}}                        & Guarkernmehl\\
        {\unit[$\nicefrac{1}{2}$]{P.}}        & Backpulver\\
        {\unit[$\nicefrac{1}{2}$]{Frucht}}    & Orangenschale\\
        {\unit[$\nicefrac{1}{2}$]{Frucht}}    & Zitronenschale\\
        {\unit[100]{g}}                       & Mandelstifte\\
        {\unit[100]{g}}                       & Rum-Rosinen\\
    }
    \preparation{
        \step Quark, Xylith, Butter und Eier vermengen; Mandeln, Guarkernmehl und Backpulver dazu geben.
        \step Zitronen- und Orangenschale, Mandelstifte und Rum-Rosinen vorsichtig unterheben.
        \step Stollenform fetten und den Backofen auf {\unit[160]{\celsius}} Umluft vorheizen.
        \step Den Stollen {\unit[20]{Minuten}} mit und {\unit[30]{Minuten}} ohne die Stollenform backen.
    }
\end{recipe}
\end{document}
