\makeatletter
\def\input@path{{../../}}
\makeatother
\providecommand{\main}{../../}
\documentclass[MeineRezepte.tex]{subfiles}
\begin{document}
\begin{recipe}[
    preparationtime = {\unit[5]{Min}},
    bakingtime = {\unit[15]{Min}},
    bakingtemperature = {\unit[160]{\celsius}},
    portion = {\portion[Blech]{1}},
    source = Lili
]{Ketogener Flammkuchen}\index{Ketogener Flammkuchen}\index{Flammkuchen}
    \graph{
        big=KetogenerFlammkuchen_0
    }
    \ingredients{
        Teig:                             & \\
        {\unit[200]{g}}                   & Frischkäse, körnig\\
        {\unit[200]{g}}                   & Gouda, gerieben\\
        3                                 & Eier\\
        {\unit[30]{g}}                    & Mandelmehl\\
        Belag:                            & \\
        {\unit[1]{B.}}                    & Creme Fraîche\\
        {\unit[$\nicefrac{1}{2}$]{B.}}    & Sahne\\
        1                                 & Zwiebel\\
                                          & Speck\\
                                          & Salz\\
                                          & Pfeffer
    }
    \preparation{
        \step Alle Teig-Zutaten vermengen und auf einem richtig gut eingebutterten Backblech verteilen.
        \step Etwa 15~Minuten bei {\unit[160 - 180]{\celsius}} vorbacken, bis er goldgelb ist.
        \step Creme Fraîche und Sahen vermischen, würzen und auf dem Teig verteilen.
        \step Die Zwiebel in halbe Ringe und den Speck in Scheiben schneiden, auf dem Teig verteilen und backen.
    }
    \hint{%
        Mit 3 Zwiebeln, Speck und Kümmel kann man auch Zwiebelkuchen machen.
    }
\end{recipe}
\end{document}
