\documentclass[MeineRezepte.tex]{subfiles}
\begin{document}
\begin{recipe}[
    preparationtime = {\unit[20]{Min}},
    portion = {\portion[l]{3}},
    source = Lili
]{Kinderpunsch}\index{Kinderpunsch}
    \graph{
        small=Kinderpunsch_1,
        big=Kinderpunsch_0
    }
    \ingredients{
        {\unit[2]{l}}                     & Wasser\\
        4                                 & Früchteteebeutel\\
        {\unit[500]{ml}}                  & Traubensaft, rot\\
        {\unit[500]{ml}}                  & Orangen- oder Apfelsaft\\
        {\unit[1]{TL}}                    & Zimt\\
        {\unit[1]{Msp.}}                  & Nelke\\
        {\unit[$\nicefrac{1}{2}$]{EL}}    & Vanillezucker
    }
    \preparation{
        \step Wasser zum Kochen bringen und in einem großen Topf einen starken Tee zubereiten.
        \step Den Saft dazu geben, kurz aufkochen lassen und mit den Gewürzen abschmecken.
    }
    \hint{%
        Für einen süßeren Punsch mehr Saft und evt. Honig dazugeben.
    }
\end{recipe}
\end{document}
