\documentclass[MeineRezepte.tex]{subfiles}
\begin{document}
\begin{recipe}[
    preparationtime={\unit[15]{Min}},
    portion = {\portion[ml]{600}},
    source = Etienne
]{Whisky-Kräuter-Likör}\index{Whisky-Kräuter-Likoer@Whisky-Kräuter-Likör}
    \graph{
        small=WhiskyKraeuterLikoer_1,
        big=WhiskyKraeuterLikoer_0
    }
    \ingredients{
        {\unit[$\nicefrac{1}{2}$]{l}}    & Whisky (\unit[46]{\%})\\
        {\unit[3]{EL}}                   & Honig\\
        {\unit[2]{Zweige}}               & Thymian\\
        1                                & Sternanis\\
        3                                & Nelken\\
        {\unit[3]{Nadeln}}               & Rosmarin\\
        {\unit[1]{Streifen}}             & Orangenschale\\
        {\unit[100]{ml}}                 & Wasser, destilliert
    }
    \preparation{
        \step Honig im Whisky auflösen und mit Thymian, Anis, Nelken und Rosmarin in eine saubere Flasche füllen.
        \step Eine Woche an einem dunklen und kühlen Ort ziehen lassen, dann die Orangenschale in die Flasche geben.
        \step Noch eine Woche ziehen lassen, dann die Gewürze absieben und mit dem Wasser verdünnen.
        \step Weiterhin kühl und dunkel lagern.
    }
    \hint.
    }
\end{recipe}
\end{document}
