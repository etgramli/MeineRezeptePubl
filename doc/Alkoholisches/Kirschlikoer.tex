\documentclass[MeineRezepte.tex]{subfiles}
\begin{document}
\begin{recipe}[
    preparationtime={\unit[30]{Min}},
    portion = {\portion[ml]{700}},
    source = Etienne
]{Kirschlikör}\index{Kirschlikoer@Kirschlikör}
    \graph{
        big=Kirschlikoer_0
    }
    \ingredients{
        {\unit[300]{g}}                  & Kirschen\\
        {\unit[$\nicefrac{1}{2}$]{l}}    & Kirschwasser (\unit[40]{\%})\\
        {\unit[1]{TL}}                   & Zimt\\
        2                                & Nelken\\
        $\nicefrac{1}{2}$                & Bio-Orange (Schale)
    }
    \preparation{
        \step Kirschen waschen, entkernen und in eine Flasche mit weitem Hals füllen. Steine aufheben und trocknen.
        \step Kirschwasser und Gewürze dazu geben und drei Wochen an einem kühlen und dunklen Ort ziehen lassen.
        \step Likör durch einen Kaffeefilter abseihen und in saubere Flaschen abfüllen.
        \step Wenn der Hals breit genug ist kann man noch ein paar Kirschen dazu geben.
    }
    \hint{%
        Kann man gut vor Weihnachten als Geschenk machen.
    }
\end{recipe}
\end{document}
