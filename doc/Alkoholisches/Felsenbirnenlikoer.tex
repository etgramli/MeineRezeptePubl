\documentclass[MeineRezepte.tex]{subfiles}
\begin{document}
\begin{recipe}[
    preparationtime={\unit[15]{Min}},
    portion = {\portion[l]{$\nicefrac{1}{2}$}},
    source = Etienne
]{Felsenbirnenlikör}\index{Felsenbirnenlikoer@Felsenbirnenlikör}
    \graph{
        big=Felsenbirnenlikoer_0
    }
    \ingredients{
        {\unit[350]{g}}                       & Felsenbirnen\\
        {\unit[50]{g}}                        & Johannisbeeren\\
        {\unit[$\nicefrac{1}{2}$]{l}}         & Wodka (\unit[40]{\%})\\
        {\unit[$\nicefrac{1}{2}$]{Schote}}    & Vanille
    }
    \preparation{
        \step Felsenbirnen und Johannisbeeren waschen, abtrocknen und einige Felsenbirnen einfrieren.
        \step Die Früchte leicht eindrücken, Vanilleschote einritzen und alles in eine {\unit[1]{l}}-Flasche füllen.
        \step Mit dem Wodka auffüllen und 6~bis~8~Wochen an einem kühlen und dunklen Ort ziehen lassen.
        \step Den Likör durch einen Kaffeefilter abseihen und mit den gefrorenen Felsenbirnen in eine saubere Flasche füllen. Weiterhin an einem dunklen Platz aufbewahren.
    }
    \hint{%
        Kann man statt mit Johannisbeeren auch mit Rhabarber machen.
    }
\end{recipe}
\end{document}
