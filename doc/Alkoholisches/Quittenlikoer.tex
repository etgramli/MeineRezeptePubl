\documentclass[MeineRezepte.tex]{subfiles}
\begin{document}
\begin{recipe}[
    preparationtime = {\unit[20]{Min}},
    portion = {\portion[l]{1}},
    source = Etienne
]{Quittenlikör}\index{Quittenlikoer@Quittenlikör}
    \graph{
        big=Quittenlikoer_0
    }
    \ingredients{
        {\unit[1]{kg}}                    & Quitten\\
        {\unit[$1\nicefrac{1}{4}$]{l}}    & Wodka (\unit[40]{\%})\\
        2                                 & Zimtstangen\\
        1                                 & Nelke\\
        oder                              & \\
        {\unit[2]{TL}}                    & Vanille\\
        $\nicefrac{1}{2}$                 & Bio-Zitrone (Schale)
    }
    \preparation{
        \step Die Quitten waschen, in grobe Würfel schneiden und in einen Topf füllen.
        \step Entweder Zimt und Nelke, oder Vanille und Zitronenschale dazu geben.
        \step Wodka darauf geben, abdecken und drei Wochen an einem kühlen und dunklen Ort ziehen lassen.
        \step Die Zitronenschale nach einer Woche herausnehmen, damit der Quittengeschmack nicht untergeht.
        \step Likör durch einen Kaffeefilter abseihen und in saubere Flaschen abfüllen.
    }
    \hint{%
        Man kann auch noch etwas Orangenschale dazu tun.
    }
\end{recipe}
\end{document}
