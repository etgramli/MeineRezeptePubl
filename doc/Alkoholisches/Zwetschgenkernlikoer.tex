\documentclass[MeineRezepte.tex]{subfiles}
\begin{document}
\begin{recipe}[
    preparationtime = {\unit[15]{Min}},
    portion = {\portion[l]{1}},
    source = Etienne
]{Zwetschgenkernlikör}\index{Zwetschgenkernlikoer@Zwetschgenkernlikör}
    \graph{
        big=Zwetschgenkernlikoer_0
    }
    \ingredients{
        {\unit[$\nicefrac{1}{2}$]{kg}}    & Zwetschgen, nur die Kerne\\
        {\unit[700]{ml}}                  & Wodka (\unit[40]{\%})\\
        {\unit[30]{g}}                    & Zucker
    }
    \preparation{
        \step Zwetschgenkerne trocknen und mit dem Wodka in eine 1-Liter-Flasche füllen. 3-4 Monate ziehen lassen.
        \step Den Likör auf \unit[25]{\%} bis \unit[30]{\%} verdünnen und mit dem Zucker abschmecken.
        \step In saubere Flaschen umfüllen und kühl aufbewahren.
    }
    \hint{%
        Schmeckt so ähnlich wie Amaretto.\\
        Diesen Likör macht man am Besten gleichzeitig mit einem Zwetschgenkuchen oder -likör.
    }
\end{recipe}
\end{document}
