\documentclass[MeineRezepte.tex]{subfiles}
\begin{document}
\begin{recipe}[
    bakingtime = {\unit[30]{Min}},
    bakingtemperature = {\unit[180]{\celsius}},
    source = Lili
]{Rhabarberkuchen mit Hefeteig}\index{Rhabarberkuchen}
    \graph{
        small=RabarberkuchenHefeteig_1,
        big=RabarberkuchenHefeteig_0
    }
    \ingredients{
        Hefeteig:                             & \\
        {\unit[70]{ml}}                       & Milch\\
        {\unit[2]{EL}}                        & Zucker\\
        {\unit[1]{Prise}}                     & Salz\\
        {\unit[1]{EL}}                        & Butter\\
        {\unit[$\nicefrac{1}{4}$]{Würfel}}    & Hefe\\
        {\unit[150]{g}}                       & Mehl\\
                                              & \\
        Belag:                                & \\
        {\unit[3-4]{St.}}                     & Rhabarber\\
        etwas                                 & Zucker\\
                                              & Nüsse, gemahlen\\
        {\unit[1]{B.}}                        & Schmand\\
        2                                     & Eier\\
        {\unit[1]{P.}}                        & Vanille-Zucker\\
        {\unit[2]{EL}}                        & Zucker\\
                                              & Zimt
    }
    \preparation{
        \step Hefeteig nach Gehen in Springform, nochmals gehen lassen, mit gemahlenen Nüssen bestreuen.
        \step Rhabarber schneiden und auf den Teig geben; darüber die Masse aus Schmand, Eiern etc.
        \step Im vorgeheizten Ofen bei {\unit[180]{\celsius}} ca. 30 Minuten in der Mitte backen.
    }
\end{recipe}
\end{document}
