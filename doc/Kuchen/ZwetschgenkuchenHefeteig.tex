\makeatletter
\def\input@path{{../../}}
\makeatother
\providecommand{\main}{../../}
\documentclass[MeineRezepte.tex]{subfiles}
\begin{document}
\begin{recipe}[
    bakingtime = {\unit[30]{Min}},
    bakingtemperature = {\unit[180]{\celsius}},
    source = Lili
]{Zwetschgenkuchen (Hefeteig)}\index{Zwetschgenkuchen}
    \graph{
        big=ZwetschgenkuchenHefeteig_0
    }
    \ingredients{
        {\unit[500]{g}}                       & Zwetschgen\\
        {\unit[2]{EL}}                        & Semmelbrösel\\
                                              & \\
        Boden:                                & \\
        {\unit[150]{g}}                       & Mehl\\
        {\unit[70]{ml}}                       & Milch, lauwarm\\
        {\unit[2]{EL}}                        & Zucker\\
        {\unit[1]{Prise}}                     & Salz\\
        {\unit[1]{EL}}                        & Butter\\
        {\unit[$\nicefrac{1}{4}$]{Würfel}}    & Hefe\\
                                              & \\
        Streusel:                             & \\
        {\unit[40]{g}}                        & Zucker\\
        {\unit[40]{g}}                        & Butter\\
        {\unit[60]{g}}                        & Mehl\\
        {\unit[$\nicefrac{1}{2}$]{TL}}        & Zimt
    }
    \preparation{
        \step Hefeteig zubereiten und gehen lassen; die Zwetschgen entsteinen und vierteln.
        \step Teig in eine gefettete Springform kneten, mit Semmelbröseln bestreuen.
        \step Die Zwetschgen dachziegelartig auf dem Teig verteilen; die Streusel zubereiten und darauf streuen.
        \step Im vorgeheizten Backofen in der Mitte 25-30 Minuten bei {\unit[180]{\celsius}} backen.
    }
\end{recipe}
\end{document}
