\makeatletter
\def\input@path{{../../}}
\makeatother
\providecommand{\main}{../../}
\documentclass[MeineRezepte.tex]{subfiles}
\begin{document}
\begin{recipe}[
    preparationtime = {\unit[30]{Min.}},
    source = Lili
]{Schwarzwälder Kirschtorte}\index{Schwarzwaelder Kirschtorte@Schwarzwälder Kirschtorte}
    \graph{
        big=SchwarzwaelderKirschtorte_0
    }
    \ingredients{
        5                                & Eier\\
        {\unit[180]{g}}                  & Zucker\\
        {\unit[1]{P.}}                   & Vanille-Zucker\\
        {\unit[100]{g}}                  & Mehl\\
        {\unit[3]{TL}}                   & Backpulver\\
        {\unit[50]{g}}                   & Speisestärke\\
        {\unit[50]{g}}                   & Kakao\\
        Belag:                           & \\
        {\unit[460]{g}}                  & Sauerkirschen\\
        {\unit[$\nicefrac{1}{4}$]{l}}    & Kirschsaft\\
        {\unit[1]{Msp.}}                 & Zimt\\
        {\unit[3]{EL}}                   & Stärke\\
        {\unit[4]{B.}}                   & Sahne\\
        {\unit[40]{ml}}                  & Kirschwasser\\
        16                               & Kirschen\\
                                         & Schokoraspel
    }
    \preparation{
        \step Zwei Tage vorher Biskuitboden machen; ein Tag vorher zwei mal durchschneiden.
        \step Saft mit Zimt aufkochen, Stärke mit Wasser oder Saft verrühren und zum Kochenden geben.
        \step Einige Male aufkochen lassen, dann Kirschen dazu tun und abkühlen lassen.
        \step Masse auf den ersten Boden streichen; den zweiten Boden darauf legen und mit Kirschwasser benetzen.
        \step Den zweiten Boden mit geschlagener Sahne bestreichen und den dritten darauf legen.
        \step Den Kuchen mit Sahne, Kirschen und Schokoraspel garnieren.
    }
\end{recipe}
\end{document}
