\makeatletter
\def\input@path{{../../}}
\makeatother
\providecommand{\main}{../../}
\documentclass[MeineRezepte.tex]{subfiles}
\begin{document}
\begin{recipe}[
    bakingtime = {\unit[35]{Min}},
    bakingtemperature = {\unit[180]{\celsius}},
    source = Lili
]{Schneewittchenkuchen}\index{Schneewittchenkuchen}
    \graph{
        small=Schneewittchenkuchen_1,
        big=Schneewittchenkuchen_0
    }
    \ingredients{
        3                                  & Eigelb\\
        {\unit[100]{g}}                    & Butter\\
        {\unit[150]{g}}                    & Zucker\\
        {\unit[1]{P.}}                     & Vanille-Zucker\\
        {\unit[200]{g}}                    & Mehl\\
        {\unit[$2\nicefrac{1}{2}$]{TL}}    & Backpulver\\
        {\unit[2]{EL}}                     & Milch\\
        {\unit[1]{EL}}                     & Kakao\\
        {\unit[1]{Glas}}                   & Sauerkirschen\\
        {\unit[250]{g}}                    & Quark\\
        {\unit[1]{P.}}                     & Vanille-Zucker\\
        {\unit[$\nicefrac{1}{4}$]{l}}      & Sahne, süß\\
        {\unit[2]{P.}}                     & Sahnesteif\\
        {\unit[2]{EL}}                     & Puderzucker\\
        {\unit[1]{P.}}                     & Tortenguss\\
        {\unit[$\nicefrac{1}{4}$]{l}}      & Kirschsaft
    }
    \preparation{
        \step Teig für Boden wie Marmorkuchen zubereiten, in Springform verteilen, Kirschen darauf legen.
        \step Im vorgeheizten Backofen 35 Minuten bei {\unit[180]{\celsius}} backen und abkühlen lassen.
        \step Sahne steif schlagen, mit Quark und Vanille-Zucker vermischen und auf dem Boden verstreichen.
        \step Tortenguss zubereiten und noch heiß auf den Kuchen geben; 2 Stunden kühl stellen.
    }
\end{recipe}
\end{document}
