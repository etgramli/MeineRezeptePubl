\makeatletter
\def\input@path{{../../}}
\makeatother
\providecommand{\main}{../../}
\documentclass[MeineRezepte.tex]{subfiles}
\begin{document}
\begin{recipe}[
    bakingtime = {\unit[1]{h}},
    bakingtemperature = {\unit[180]{\celsius}},
    source = Etienne
]{Schoko-Nuss-Kuchen}\index{Schoko-Nuss-Kuchen}
    \graph{
        big=SchokoNussKuchen_0
    }
    \ingredients{
        {\unit[250]{g}}                   & Mehl\\
        {\unit[250]{g}}                   & Haselnüsse\\
        {\unit[150]{g}}                   & Zucker\\
        {\unit[1]{EL}}                    & Vanillezucker\\
        {\unit[1]{P.}}                    & Backpulver\\
        {\unit[2]{EL}}                    & Kakaopulver\\
        {\unit[$\nicefrac{1}{4}$]{TL}}    & Zimt\\
        {\unit[80-100]{ml}}               & Kaffee / Espresso\\
        {\unit[200]{g}}                   & Sojasahne\\
        1                                 & Karotte\\
        {\unit[150]{g}}                   & Bitterschokolade
    }
    \preparation{
        \step Die Karotte schälen, raspeln und mit den trockenen Zutaten verrühren.
        \step Dann Kaffee und Sojasahne nach und nach beim Mixen dazu geben.
        \step Alles in eine Springform geben und ca. eine Stunde bei {\unit[180]{\celsius}} im Ofen backen.
        \step Die Schokolade schmelzen und auf dem Kuchen verteilen; vor dem Essen abkühlen lassen.
    }
    \hint{%
        Mein erster veganer Kuchen (außer dass ich die Form mit Butter eingefettet habe).
    }
\end{recipe}
\end{document}
