\makeatletter
\def\input@path{{../../}}
\makeatother
\providecommand{\main}{../../}
\documentclass[MeineRezepte.tex]{subfiles}
\begin{document}
\begin{recipe}[
    bakingtime = {\unit[35-45]{Min}},
    bakingtemperature = {\unit[180]{\celsius}},
    source = Lili
]{Schoko-Bohnen-Kuchen}\index{Schoko-Bohnen-Kuchen}
    \graph{
        big=SchokoBohnenKuchen_0
    }
    \ingredients{
        {\unit[1]{kl. Dose}}    & Kidneybohnen\\
        6                       & Eier\\
        {\unit[2]{TL}}          & Vanille-Zucker\\
        {\unit[125]{g}}         & Butter\\
        {\unit[150]{g}}         & Xylith\\
        {\unit[4]{EL}}          & Kakao\\
        {\unit[3-4]{EL}}        & Nüsse gehackt u. geröstet\\
        {\unit[1]{Prise}}       & Zimt\\
        {\unit[1]{Prise}}       & Nelke\\
        {\unit[1]{Prise}}       & Ingwer\\
        {\unit[2]{gestr. TL}}   & Backpulver
    }
    \preparation{
        \step Kidneybohnen abwaschen und mit einem Ei und {\unit[2]{TL}} Vanille-Zucker vermischen.
        \step Butter und Xylith 30~Sekunden bei Stufe~4,5 schaumig rühren.
        \step 5 Eier dazu geben und 20~Sekunden bei Stufe~4,5 vermischen.
        \step Bohnenmasse dazu geben und 20~Sekunden bei Stufe~4 vermischen.
        \step Kakao, Nüsse, Zimt, Nelke, Ingwer und Backpulver dazu geben und 20~Sekunden bei Stufe~4 vermischen.
        \step In eine Backform geben und {\unit[35-45]{Minuten}} bei {\unit[180]{\celsius}} backen.
    }
\end{recipe}
\end{document}
