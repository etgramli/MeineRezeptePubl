\documentclass[MeineRezepte.tex]{subfiles}
\begin{document}
\begin{recipe}[
    bakingtime = {\unit[45]{Min}},
    bakingtemperature = {\unit[180]{\celsius}},
    source = Etienne
]{Whisky-Schokoladen-Kuchen}\index{Whisky-Schokoladen-Kuchen}
    \graph{
        small=WhiskySchokoladenKuchen_1,
        big=WhiskySchokoladenKuchen_0
    }
    \ingredients{
        5                   & Eier\\
        {\unit[150]{g}}     & Zucker\\
        {\unit[250]{g}}     & Butter\\
        {\unit[2]{P.}}      & Vanille-Zucker\\
        {\unit[1]{TL}}      & Zimt\\
        {\unit[1]{TL}}      & Kakao\\
        {\unit[250]{g}}     & Mehl\\
        {\unit[120]{g}}     & Rosinen\\
        {\unit[160]{ml}}    & Whisky, sherryfassgereift\\
        {\unit[1]{P.}}      & Backpulver\\
        {\unit[250]{g}}     & Schokolade
    }
    \preparation{
        \step Die Rosinen im Whisky einlegen, bis sie vollgesogen sind.
        \step Butter, Zucker, Vanille-Zucker und Eier schaumig rühren.
        \step Die restlichen Zutaten, bis auf die Schokolade, unterrühren und in eine gut gefettete Springform füllen.
        \step Bei {\unit[180]{\celsius}} im vorgeheizten Backofen ca. 45 Minuten backen.
        \step Schokolade mit etwas Butter bei ca. {\unit[50]{\celsius}} schmelzen und den leicht abgekühlten Kuchen glasieren.
    }
    \hint{%
        Der Whisky sollte nicht zu würzig sein: Glenfarclas 12y passt gut, Bunnahabhain 12y eher nicht.
    }
\end{recipe}
\end{document}
