\documentclass[MeineRezepte.tex]{subfiles}
\begin{document}
\begin{recipe}[
    bakingtime = {\unit[40]{Min}},
    bakingtemperature = {\unit[200]{\celsius}},
    source = Oma Rita
]{Rahmkuchen}\index{Rahmkuchen}
    \graph{
        small=Rahmkuchen_1,
        big=Rahmkuchen_0
    }
    \ingredients{
        Boden:                                & \\
        {\unit[150-200]{g}}                   & Mehl\\
        {\unit[70]{ml}}                       & Milch, lauwarm\\
        {\unit[2]{EL}}                        & Zucker\\
        {\unit[1]{Prise}}                     & Salz\\
        {\unit[1]{EL}}                        & Butter\\
        {\unit[$\nicefrac{1}{4}$]{Würfel}}    & Hefe\\
                                              & \\
        Belag:                                & \\
        {\unit[250]{ml}}                      & Sahne\\
        {\unit[$\nicefrac{1}{2}$]{P.}}        & Creme"-torten"-pul"-ver\\
        {\unit[2-3]{EL}}                      & Zucker\\
        2                                     & Eier\\
        {\unit[200-250]{g}}                   & Sahnequark
    }
    \preparation{
        \step Den Hefeteig zubereiten und in eine Springform aus-gewellten.
        \step Zutaten für den Belag vermengen und auf den Teig leeren; etwas Zucker und Zimt darüber streuen.
        \step Im vorgeheizten Backofen 30-40 Minuten bei {\unit[200]{\celsius}} backen bis der Kuchen goldbraun ist.
    }
    \hint{%
        Sehr fein auch, wenn man dünne Apfelscheiben auf den Teig legt.
    }
\end{recipe}
\end{document}
