\documentclass[MeineRezepte.tex]{subfiles}
\begin{document}
\begin{recipe}[
    bakingtime = {\unit[25-30]{Min}},
    bakingtemperature = {\unit[180]{\celsius}},
    source = Lili
]{Linzer Torte}\index{Linzer Torte}
    \graph{
        big=LinzerTorte_0
    }
    \ingredients{
        {\unit[170]{g}}                   & Haselnüsse\\
        {\unit[150]{g}}                   & Mehl\\
        {\unit[60]{g}}                    & Zucker\\
        1                                 & Ei\\
        {\unit[1]{EL}}                    & Vanillezucker\\
        {\unit[1]{TL}}                    & Zimt\\
        {\unit[$\nicefrac{1}{2}$]{TL}}    & Nelken\\
        {\unit[1]{TL}}                    & Backpulver\\
        {\unit[100]{g}}                   & Butter, kalt\\
        {\unit[300]{g}}                   & Himbeermarmelade
    }
    \preparation{
        \step Haselnüsse rösten (höchstens {\unit[150]{\celsius}}), bis sie duften, abkühlen lassen und dann mahlen.
        \step Mehl, Zucker, Ei, Gewürze, Butter und Nüsse zu einem glatten Teig kneten; bei Bedarf kalt stellen.
        \step $\nicefrac{2}{3}$ des Teigs in einer Springform ausrollen. Den Rest in gleichmäßige Streifen schneiden.
        \step Konfitüre auf dem Boden verstreichen, die Streifen gitterförmig darauf verteilen.
        \step Den Kuchen 25 bis 30 Minuten bei {\unit[180]{\celsius}} backen.
    }
    \hint{%
        Mit einer Frischhaltefolie kann man den Teig einfach, ohne kleben auf dem Boden verteilen.
    }
\end{recipe}
\end{document}
