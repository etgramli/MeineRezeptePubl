\documentclass[MeineRezepte.tex]{subfiles}
\begin{document}
\begin{recipe}[
    preparationtime = {\unit[20]{Min}},
    bakingtime = {\unit[1]{h}},
    bakingtemperature = {\unit[175]{\celsius}},
    portion = {\portion{12}},
    source = Lili
]{Rüblekuchen}\index{Rueblekuchen@Rüblekuchen}
    \graph{
        small=Rueblekuchen_1,
        big=Rueblekuchen_0
    }
    \ingredients{
        6                                 & Eier\\
        {\unit[160]{g}}                   & Zucker\\
        {\unit[300]{g}}                   & Möhren, gerieben\\
        {\unit[300]{g}}                   & Mandeln, gerieben\\
        {\unit[40]{g}}                    & Stärkemehl\\
        {\unit[$\nicefrac{1}{2}$]{TL}}    & Zimt\\
                                          & Zitronensaft (Bio)\\
                                          & Zitronenschale (Bio)\\
                                          & Puderzucker\\
                                          & Marzipan-Möhren
    }
    \preparation{
        \step Das Eiweiß in hohem Gefäß zu steifem Schnee schlagen und mit der Hälfte der Mandeln vermischen.
        \step Das Eigelb mit Zucker schaumig schlagen, die geriebenen Möhren dazu geben und gut unter mischen.
        \step Die Hälfte der Mandeln, Stärke, Zimt, Zitronensaft und -schale einrühren.
        \step Das Eiklar unter die Masse heben und in eine eingefettete Springform geben.
        \step Im vorgeheizten Backofen eine Stunde bei {\unit[175]{\celsius}} backen, danach auf einem Gitter auskühlen lassen.
        \step Nach Belieben mit Marzipan-Möhren und Puderzucker dekorieren.
    }
    \hint{%
        Die Möhren kann man in Stücken im Thermomix 1 Sekunden bei Stufe 6 zerkleinern.\\
        Man kann auch einen Zitronenguss aus Zitronensaft und Puderzucker machen.
    }
\end{recipe}
\end{document}
