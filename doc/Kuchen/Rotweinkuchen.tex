\makeatletter
\def\input@path{{../../}}
\makeatother
\providecommand{\main}{../../}
\documentclass[MeineRezepte.tex]{subfiles}
\begin{document}
\begin{recipe}[
    bakingtime = {\unit[45]{Min}},
    bakingtemperature = {\unit[180]{\celsius}},
    source = Lili
]{Weltuntergangskuchen}\index{Weltuntergangskuchen}\index{Rotweinkuchen}
    \graph{
        big=Rotweinkuchen_0
    }
    \ingredients{
        5                      & Eier\\
        {\unit[150-180]{g}}    & Zucker\\
        {\unit[250]{g}}        & Butter\\
        {\unit[2]{P.}}         & Vanille-Zucker\\
        {\unit[2]{TL}}         & Zimt und Kakao\\
        {\unit[100]{ml}}       & Rotwein\\
        {\unit[250-280]{g}}    & Mehl\\
        {\unit[1]{Glas}}       & Kirschen\\
        {\unit[1]{P.}}         & Backpulver\\
        {\unit[200]{g}}        & Schokolade
    }
    \preparation{
        \step Butter, Zucker, Vanille-Zucker und Eier schaumig rühren.
        \step Zimt, Kakao, Rotwein, Mehl und Backpulver dazu rühren.
        \step Die Springform gut fetten und den Teig mit den Kirschen hinein füllen.
        \step Bei {\unit[180]{\celsius}} im vorgeheizten Backofen ca. 45 Minuten backen.
        \step Schokolade mit etwas Butter bei ca. {\unit[50]{\celsius}} schmelzen und den leicht abgekühlten Kuchen glasieren.
    }
    \hint{%
        2012: Angeblich Weltuntergang, den Kuchen mehrfach gebacken -> Weltuntergangskuchen\newline
        Wir tun noch Kirschen dazu. Sauerkirschen. Abgetropfte Sauerkirschen.\newline
        Wird in einer Silikonform sehr kompakt!
    }
\end{recipe}
\end{document}
