\makeatletter
\def\input@path{{../../}}
\makeatother
\providecommand{\main}{../../}
\documentclass[MeineRezepte.tex]{subfiles}
\begin{document}
\begin{recipe}[
    preparationtime = {\unit[10]{Min}},
    bakingtime = {\unit[35]{Min}},
    bakingtemperature = {\unit[200]{\celsius}},
    portion = {\portion[Portionen]{6}},
    source = Etienne
]{Drunken Crumble}\index{Drunken Crumble}
    \graph{
        big=DrunkenCrumble_0
    }
    \ingredients{
        Füllung:                          & \\
        {\unit[600]{g}}                   & Rhabarber / Äpfel\\
        {\unit[120]{ml}}                  & Whisky\\
        {\unit[100]{g}}                   & Zucker, braun\\
        {\unit[1]{Msp.}}                  & Nelke\\
        {\unit[$\nicefrac{1}{2}$]{TL}}    & Zimt\\
                                          & Orangen- \& Zitronenschale, gerieben\\
                                          & \\
        Streusel:                         & \\
        {\unit[180]{g}}                   & Mehl\\
        {\unit[90]{g}}                    & Butter\\
        {\unit[90]{g}}                    & Zucker\\
        {\unit[1]{Msp.}}                  & Kardamom\\
        {\unit[$\nicefrac{1}{2}$]{TL}}    & Zimt\\
                                          & Orangen- \& Zitronenschale, gerieben
    }
    \preparation{
        \step Rhabarber waschen, in Stücke schneiden und mit den anderen Füllungs-Zutaten in eine Auflaufform geben.
        \step Die Zutaten für die Streusel vermischen, bis Krümel entstehen und über die Füllung streuen.
        \step Etwa {\unit[35]{Minuten}} bei {\unit[200]{\celsius}} backen, bis die Streusel bräunen.
    }
\end{recipe}
\end{document}
