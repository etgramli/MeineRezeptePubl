\makeatletter
\def\input@path{{../../}}
\makeatother
\providecommand{\main}{../../}
\documentclass[MeineRezepte.tex]{subfiles}
\begin{document}
\begin{recipe}[
    bakingtime = {\unit[30]{Min}},
    bakingtemperature = {\unit[220]{\celsius}},
    source = Oma Tilda
]{Käsekuchen vom Blech}\index{Kaesekuchen@Käsekuchen}
    \graph{
        small=KaesekuchenVomBlech_1,
        big=KaesekuchenVomBlech_0
    }
    \ingredients{
        Mürbteig:                         & \\
        1                                 & Ei\\
        {\unit[300]{g}}                   & Mehl\\
        {\unit[150]{g}}                   & Butter\\
        {\unit[80]{g}}                    & Zucker\\
        {\unit[1]{Prise}}                 & Salz\\
        Belag:                            & \\
        1                                 & Bio-Zitrone\\
        {\unit[750]{g}}                   & Magerquark\\
        {\unit[100]{g}}                   & Butter\\
        {\unit[200]{g}}                   & Zucker\\
        4                                 & Eier\\
        {\unit[80]{g}}                    & Speisestärke\\
        {\unit[2]{Dosen}}                 & Mandarinen oder Aprikosen\\
        Streusel:                         & \\
        {\unit[300]{g}}                   & Mehl\\
        {\unit[190]{g}}                   & Butter\\
        {\unit[190]{g}}                   & Zucker\\
        {\unit[$\nicefrac{1}{2}$]{TL}}    & Zimt
    }
    \preparation{
        \step Mürbteig zubereiten, auf dem Backblech ausrollen und mit Gabel einstechen.
        \step Im vorgeheizten Backofen 10 Minuten bei {\unit[225]{\celsius}} backen, abkühlen lassen.
        \step Belag auf den Mürbteig streichen, Streusel darauf verteilen.
        \step Im vorgeheizten Backofen bei {\unit[220]{\celsius}} 20-30 Minuten backen.
    }
\end{recipe}
\end{document}
