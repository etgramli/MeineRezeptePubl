\makeatletter
\def\input@path{{../../}}
\makeatother
\providecommand{\main}{../../}
\documentclass[MeineRezepte.tex]{subfiles}
\begin{document}
\begin{recipe}[
    bakingtime = {\unit[50-60]{Min}},
    bakingtemperature = {\unit[160]{\celsius}},
    source = Lili
]{Käsekuchen - das perfekte Rezept}\index{Kaesekuchen@Käsekuchen}
    \graph{
        big=KaesekuchenDasPerfekteRezept_0
    }
    \ingredients{
        Boden:                            & \\
        {\unit[80]{g}}                    & Zucker\\
        {\unit[1]{TL}}                    & Vanillezucker\\
        {\unit[100]{g}}                   & Butter, kalt\\
        {\unit[200]{g}}                   & Mehl\\
        {\unit[$\nicefrac{1}{2}$]{TL}}    & Backpulver\\
        1                                 & Ei\\
                                          & \\
        Belag:                            & \\
        3                                 & Eier\\
        {\unit[500]{g}}                   & Magerquark\\
        {\unit[250]{g}}                   & Quark (\unit[20]{\%})\\
        {\unit[1]{P.}}                    & Vanille"-pud"-ding"-pul"-ver\\
        {\unit[6]{EL}}                    & Distelöl\\
        {\unit[$1\nicefrac{1}{2}$]{EL}}   & Zitronensaft\\
        etwas                             & Zitronenschale\\
        {\unit[180]{g}}                   & Zucker\\
        {\unit[1]{B.}}                    & Sahne
    }
    \preparation{
        \step Alle Teig-Zutaten in den Mixtopf und mit dem Spatel 40~Sekunden bei Stufe~5 vermischen.
        \step Den Teig gleichmäßig in die ungebutterte Springform drücken.
        \step Eier, Zucker und Vanille-Zucker 3~Minuten bei Stufe~4 schaumig rühren.
        \step Sahne, Quark, Öl und Zitronensaft und -schale dazu geben und nochmal so lang vermischen.
        \step Das Puddingpulver dazu und eine Minute bei Stufe~4 einrühren. Auf dem Teig verteilen.
        \step Dann 50 bis 60~Minuten bei {\unit[160]{\celsius}} backen.
    }
\end{recipe}
\end{document}
