\documentclass[MeineRezepte.tex]{subfiles}
\begin{document}
\begin{recipe}[
    preparationtime = {\unit[30]{Min}},
    bakingtime = {\unit[30]{Min}},
    bakingtemperature = {\unit[180]{\celsius}},
    source = Lili
]{Zwetschgenkuchen (Quarkölteig)}\index{Zwetschgenkuchen}
    \graph{
        big=ZwetschgenkuchenQuarkoelteig_0
    }
    \ingredients{
        Boden:                            & \\
        {\unit[75]{g}}                    & Quark\\
        {\unit[2]{EL}}                    & Milch\\
        {\unit[3]{EL}}                    & Öl\\
        {\unit[30]{g}}                    & Zucker\\
        {\unit[1]{Prise}}                 & Salz\\
        {\unit[150]{g}}                   & Mehl\\
        {\unit[$\nicefrac{1}{2}$]{P.}}    & Backpulver\\
                                          & Vanille-Zucker\\
                                          & Zitronenaroma\\
        {\unit[700]{g}}                   & Zwetschgen\\
        Streusel:                         & \\
        {\unit[40]{g}}                    & Zucker\\
        {\unit[40]{g}}                    & Butter\\
        {\unit[60]{g}}                    & Mehl\\
        {\unit[$\nicefrac{1}{2}$]{TL}}    & Zimt
    }
    \preparation{
        \step Die Zutaten für den Boden $1\nicefrac{1}{2}$~Minuten bei Getreidestufe kneten; in gebutterte Springform drücken.
        \step Die Zwetschgen vierteln und dachziegelartig auf dem Boden verteilen.
        \step Die Streusel-Zutaten im Mixtopf 10~Sekunden, Stufe~7 vermischen und auf den Zwetschgen verteilen.
        \step Im vorgeheizten Backofen in der Mitte ca. 30~Minuten bei {\unit[180]{\celsius}} backen.
    }
    \hint{%
        Teig braucht nicht manuell geknetet werden; kann gleich in die Form gedrückt werden.
    }
\end{recipe}
\end{document}
