\documentclass[MeineRezepte.tex]{subfiles}
\begin{document}
\begin{recipe}[
    preparationtime = {\unit[15]{Min}},
    bakingtime = {\unit[75]{Min}},
    source = Etienne
]{Chili-Ketchup}\index{Chili-Ketchup}\index{Ketchup!Chili-Ketchup}
    \graph{
        big=ChiliKetchup_0
    }
    \ingredients{
        {\unit[700]{g}}     & Passierte Tomaten\\
        2                   & Paprika, rot\\
        2                   & Zwiebeln\\
        1                   & Chili, scharf\\
        1                   & Lorbeerblatt\\
        {\unit[75]{ml}}     & Essig\\
        {\unit[1]{TL}}      & Sojasauce\\
        {\unit[2]{TL}}      & Kurkuma\\
        {\unit[1]{TL}}      & Curry\\
        {\unit[1]{Msp.}}    & Zimt\\
        {\unit[2]{TL}}      & Jo"-han"-nis"-brot"-kern"-mehl\\
        {\unit[1]{EL}}      & Petersilie\\
        {\unit[1]{TL}}      & Oregano\\
        {\unit[1]{TL}}      & Honig\\
        {\unit[1]{TL}}      & Salz
    }
    \preparation{
        \step Paprika und Chili putzen, entkernen und mit Zwiebel in Würfel schneiden und in Öl andünsten.
        \step Tomaten, Lorbeerblatt, Essig, Sojasauce, Kurkuma, Curry und Zimt dazu geben.
        \step Kurz aufkochen, 15~Minuten köcheln lassen und dann die Masse pürieren.
        \step Bei mittlerer Hitze etwa eine Stunde köcheln lassen, oder im Thermomix 20 Minuten bei Varoma.
        \step Falls der Ketchup immer noch zu dünnflüssig ist, kann man ihn mit Johannisbrotkernmehl nachdicken.
        \step Petersilie, Oregano und Honig unterheben und mit Salz abschmecken.
        \step In eine saubere Flasche füllen und kühl lagern.
    }
    \hint{%
        Schmeckt gut zu Grillwurst und Hackfleischküchle, passt aber auch zu Weißwürsten.
    }
\end{recipe}
\end{document}
