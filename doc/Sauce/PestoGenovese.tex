\makeatletter
\def\input@path{{../../}}
\makeatother
\providecommand{\main}{../../}
\documentclass[MeineRezepte.tex]{subfiles}
\begin{document}
\begin{recipe}[
    preparationtime = {\unit[8]{Min}},
    portion = {\portion[Gläschen]{2}},
    source = Etienne
]{Pesto Genovese}\index{Pesto Genovese}\index{Pesto!Pesto Genovese}
    \graph{
        small=PestoGenovese_1,
        big=PestoGenovese_0
    }
    \ingredients{
        {\unit[80]{g}}                    & Basilikumblätter\\
        {\unit[50]{g}}                    & Parmesan\\
        {\unit[30]{g}}                    & Pinienkerne\\
        {\unit[$\nicefrac{1}{2}$]{TL}}    & Salz\\
        {\unit[150]{g}}                   & Olivenöl\\
        1                                 & Knoblauchzehe, klein
    }
    \preparation{
        \step Die Pinienkerne anrösten, mit Käse in den Mixtopf geben und 5~Sekunden auf Stufe~8 zerkleinern.
        \step Die restlichen Zutaten dazu geben und 20 Sekunden auf Stufe 7 zerkleinern.
        \step Auf Brot oder Nudeln zügig servieren oder in kleine Gläschen mit etwas Öl darauf abfüllen.
    }
    \hint{%
        Kann man in kleinen Gläschen gut einfrieren.\newline
        Statt frischem Basilikum kann man auch gefrorenen verwenden, dann wird das Pesto milchig.\newline
        Man kann es auch aus anderen Kräutern und Kernen herstellen; mit nur Bärlauch wird das Pesto sehr intensiv, dazu passen Walnusskerne.
    }
\end{recipe}
\end{document}
