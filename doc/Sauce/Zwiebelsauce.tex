\makeatletter
\def\input@path{{../../}}
\makeatother
\providecommand{\main}{../../}
\documentclass[MeineRezepte.tex]{subfiles}
\begin{document}
\begin{recipe}[
    preparationtime = {\unit[15]{Min}},
    bakingtime = {\unit[20]{Min}},
    portion = {\portion[Portionen]{2}},
    source = Etienne
]{Zwiebel-Pfeffer-Sauce}\index{Zwiebel-Pfeffer-Sauce}
    \graph{
        small=ZwiebelPfefferSauce_1,
        big=ZwiebelPfefferSauce_0
    }
    \ingredients{
        3                    & Zwiebeln, groß\\
        {\unit[2]{EL}}       & Tomatenmark\\
        {\unit[100]{ml}}     & Whisky\\
        {\unit[500]{ml}}     & Brühe\\
        {\unit[2]{EL}}       & Pfefferkörner, grün\\
        1                    & Lorbeerblatt\\
                             & Salz\\
                             & Pfeffer, schwarz
    }
    \preparation{
        \step Die Zwiebeln in (halbe) Ringe schneiden und in Öl glasig dünsten.
        \step Das Tomatenmark dazu geben und kurz anbraten, es darf aber nicht bräunen.
        \step Whisky dazu geben und anzünden, damit der Alkohol verbrennt, dann mit der Brühe ablöschen.
        \step Mit dem Lorbeer und grünem Pfeffer aufkochen und etwa {\unit[20]{Minuten}} reduzieren, bis die Sauce sämig ist.
        \step Das Lorbeerblatt herausnehmen und mit Salz und Pfeffer abschmecken.
    }
\end{recipe}
\end{document}
