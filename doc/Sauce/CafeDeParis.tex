\documentclass[MeineRezepte.tex]{subfiles}
\begin{document}
\begin{recipe}[
    preparationtime = {\unit[50]{Min}},
    portion = {\portion{4}},
    source = Etienne
]{Café de Paris}\index{Cafe de Paris@Café de Paris}
    \graph{
        small=CafeDeParis_1,
        big=CafeDeParis_0
    }
    \ingredients{
        3                    & Schalotten\\
        {\unit[1]{Zweig}}    & Thymian, frisch\\
        {\unit[1]{Zweig}}    & Rosmarin, frisch\\
        {\unit[1]{EL}}       & Tomatenmark\\
        {\unit[100]{ml}}     & Weinbrand / Whisky, trocken\\
        {\unit[50]{ml}}      & Sherry, süß / Port\\
        {\unit[500]{ml}}     & Sahne\\
                             & Brühe\\
                             & Salz\\
                             & Pfeffer
    }
    \preparation{
        \step Zwiebeln fein hacken, in Butter dünsten, Rosmarin und Thymian dazu und ebenfalls kurz anschwitzen.
        \step Tomatenmark dazu geben, rösten und mit dem Weinbrand und Sherry ablöschen.
        \step Die Sauce reduzieren, bis fast keine Flüssigkeit mehr vorhanden ist.
        \step Die Hitze reduzieren, mit Sahne ablöschen und etwa 25 Minuten köcheln lassen.
        \step Abkühlen lassen und mit dem Pürierstab mixen; falls sie zu dickflüssig ist, kann man mit Brühe verdünnen.
        \step Mit Salz und Pfeffer abschmecken und noch warm servieren.
    }
    \hint{%
        Sie passt sehr gut zu gebratenem oder gegrilltem Fleisch und Kartoffeln.\\
        Abgekühlt wird sie sehr steif.
    }
\end{recipe}
\end{document}
