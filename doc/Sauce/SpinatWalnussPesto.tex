\makeatletter
\def\input@path{{../../}}
\makeatother
\providecommand{\main}{../../}
\documentclass[MeineRezepte.tex]{subfiles}
\begin{document}
\begin{recipe}[
    preparationtime = {\unit[8]{Min}},
    portion = {\portion{4}},
    source = Etienne
]{Spinat-Walnuss-Pesto}\index{Spinat-Walnuss-Pesto}\index{Pesto!Spinat-Walnuss-Pesto}
    \graph{
        small=SpinatWalnussPesto_1,
        big=SpinatWalnussPesto_0
    }
    \ingredients{
        {\unit[80]{g}}     & Walnusskerne\\
        {\unit[200]{g}}    & Spinat\\
        {\unit[50]{g}}     & Öl\\
        {\unit[1]{TL}}     & Salz\\
                           & Pfeffer
    }
    \preparation{
        \step Die Walnusskerne anrösten, im Mixtopf 10 Sekunden auf Stufe 10 zerkleinern und umfüllen.
        \step Den Spinat im Öl pürieren, Walnüsse und Salz dazu geben und vermischen. Mit Pfeffer abschmecken.
        \step Auf Brot oder Nudeln zügig servieren oder in kleine Glasflaschen mit etwas Öl darauf abfüllen.
    }
    \hint{%
        Kann man auch mit etwas Ricotta verfeinern.
    }
\end{recipe}
\end{document}
