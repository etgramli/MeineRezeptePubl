\makeatletter
\def\input@path{{../../}}
\makeatother
\providecommand{\main}{../../}
\documentclass[MeineRezepte.tex]{subfiles}
\begin{document}
\begin{recipe}[
    preparationtime = {\unit[10]{Min}},
    portion = {\portion[Portionen]{3}},
    source = Etienne
]{Whisky-Sahne-Sauce}\index{Whisky-Sahne-Sauce}
    \graph{
        big=WhiskySauce_0
    }
    \ingredients{
        {\unit[250]{ml}}    & Sahne\\
        {\unit[50]{ml}}     & Whisky, sherryfassgereift\\
                            & Salz\\
                            & Pfeffer\\
                            & Muskat
    }
    \preparation{
        \step Den Whisky in einer Pfanne anzünden, um den Alkohol zu verbrennen.
        \step Die Sahne dazu geben und auf etwa die Hälfte einkochen.
        \step Mit Salz, viel Pfeffer und Muskat abschmecken.
    }
    \hint{%
        Passt gut zu gebratenen Maultaschen und Schweine-Medaillons.\newline
        Kann auch statt mit Muskat mit (ganzem) grünen/rosa Pfeffer und Orangenschale verfeinert werden.
    }
\end{recipe}
\end{document}
