\makeatletter
\def\input@path{{../../}}
\makeatother
\providecommand{\main}{../../}
\documentclass[MeineRezepte.tex]{subfiles}
\begin{document}
\begin{recipe}[
    preparationtime = {\unit[15]{Min}},
    portion = {\portion[Gläser]{2}},
    source = Etienne
]{Walnuss-Tomaten-Pesto}\index{Walnuss-Tomaten-Pesto}\index{Pesto!Walnuss-Tomaten-Pesto}
    \graph{
        small=WalnussTomatenPesto_1,
        big=WalnussTomatenPesto_0
    }
    \ingredients{
        {\unit[100]{g}}    & Walnusskerne\\
        {\unit[50]{g}}     & Parmesan\\
        {\unit[200]{g}}    & Getrocknete Tomaten, in Öl eingelegt\\
        1                  & Knoblauchzehe\\
        {\unit[1]{EL}}     & Balsamico-Essig\\
        {\unit[150]{g}}    & Olivenöl\\
                           & Salz\\
                           & Pfeffer
    }
    \preparation{
        \step Die Walnusskerne anrösten und mit dem Parmesan fein hacken.
        \step Getrocknete Tomaten und die Knoblauchzehe in feine Scheiben schneiden; zu den Walnüssen geben.
        \step Balsamico-Essig dazu geben, vermischen und mit Salz und Pfeffer abschmecken; das Öl dazu geben.
        \step Auf Nudeln oder Brot servieren; oder in abgekochte Gläschen abfüllen und mit Olivenöl bedecken.
    }
    \hint{%
        Mit etwas Nudelwasser wird es geschmeidiger.
    }
\end{recipe}
\end{document}
