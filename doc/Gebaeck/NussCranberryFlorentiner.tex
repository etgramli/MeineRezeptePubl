\makeatletter
\def\input@path{{../../}}
\makeatother
\providecommand{\main}{../../}
\documentclass[MeineRezepte.tex]{subfiles}
\begin{document}
\begin{recipe}[
    bakingtime = {\unit[10-15]{Min}},
    bakingtemperature = {\unit[170]{\celsius}},
    source = Lili
]{Nuss-Cranberry-Florentiner}\index{Nuss-Cranberry-Florentiner}
    \graph{
        big=NussCranberryFlorentiner_0
    }
    \ingredients{
        {\unit[170]{g}}          & Haselnüsse oder Mandeln \\
        {\unit[80]{g}}           & Cranberries, getrocknet \\
        {\unit[20]{g}}           & Butter \\
        {\unit[50]{g}}           & Sahne \\
        {\unit[2]{EL}}           & Zucker \\
        {\unit[1]{TL}}           & Vanille-Zucker \\
        {\unit[1]{TL}}           & Orangen-Zucker \\
        {\unit[2]{EL}}           & Mehl \\
        {\unit[1]{gestr. TL}}    & Natron \\
        {\unit[1]{Msp.}}         & Kardamon \\
        {\unit[1]{Msp.}}         & Ingwer \\
        {\unit[1-2]{EL}}         & Zitronensaft \\
        1                        & Ei(klar) \\
                                  & Zimt
    }
    \preparation{
        \step Die Nüsse ca. 20 Sekunden bei Stufe 5 hacken und dann umfüllen.
        \step Die Früchte 10-15 Sekunden bei Stufe 5 hacken und zu den Nüssen geben.
        \step Butter, Sahne und Zucker 4 Minuten bei Varoma auf Stufe 1,5 kochen.
        \step Nüsse, Früchte, Mehl, Ei und Gewürze dazu geben und bei Linkslauf, Stufe 3 ca. 30 Sekunden mischen.
        \step Kleine Häufchen auf ein Backblech legen, im vorgeheizten Backofen 10-15 Minuten bei {\unit[170]{\celsius}} backen.
    }
\end{recipe}
\end{document}
