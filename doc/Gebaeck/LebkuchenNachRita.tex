\makeatletter
\def\input@path{{../../}}
\makeatother
\providecommand{\main}{../../}
\documentclass[MeineRezepte.tex]{subfiles}
\begin{document}
\begin{recipe}[
    bakingtime = {\unit[12]{Min}},
    bakingtemperature = {\unit[180]{\celsius}},
    portion = {\portion[Bleche]{2}},
    source = Oma Rita
]{Lebkuchen nach Rita}\index{Lebkuchen nach Rita}
    \graph{
        big=LebkuchenRita_0
    }
    \ingredients{
        {\unit[350]{g}}                   & Mehl\\
        {\unit[250]{g}}                   & Zucker\\
        2                                 & Eier\\
        {\unit[$\nicefrac{1}{4}$]{TL}}    & Natron\\
        {\unit[1]{P.}}                    & Lebkuchengewürz\\
        {\unit[65]{g}}                    & Honig\\
        {\unit[$\nicefrac{1}{2}$]{P.}}    & Zitroenaroma\\
        {\unit[$\nicefrac{1}{2}$]{P.}}    & Orangenaroma\\
        {\unit[50]{g}}                    & Puderzucker\\
        {\unit[1]{EL}}                    & Wasser, heiß
    }
    \preparation{
        \step Trockene Zutaten erst vermischen, dann Eier und Honig dazu geben.
        \step So lange kneten bis geschmeidiger Teig entsteht, {\unit[3-4]{mm}} dick ausrollen und Herzen ausstechen.
        \step Mit Abstand auf ein Backblech legen; bei {\unit[180]{\celsius}} 10-12~Minuten backen.
        \step Puderzucker mit Wasser glatt rühren und auf die heißen Plätzchen pinseln.
    }
\end{recipe}
\end{document}
