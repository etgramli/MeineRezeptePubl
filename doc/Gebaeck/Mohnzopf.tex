\makeatletter
\def\input@path{{../../}}
\makeatother
\providecommand{\main}{../../}
\documentclass[MeineRezepte.tex]{subfiles}
\begin{document}
\begin{recipe}[
    bakingtime = {\unit[35]{Min}},
    bakingtemperature = {\unit[170]{\celsius}},
    source = Lili
]{Mohnzopf}\index{Mohnzopf}
    \graph{
        big=Mohnzopf_0,
    }
    \ingredients{
        Teig:                             & \\
        {\unit[375]{g}}                   & Mehl\\
        {\unit[20]{g}}                    & Hefe\\
        {\unit[125]{ml}}                  & Milch, lauwarm\\
        {\unit[1]{P.}}                    & Vanille-Zucker\\
        {\unit[50]{g}}                    & Zucker\\
        {\unit[50]{g}}                    & Butter, warm\\
        1                                 & Ei\\
        {\unit[$\nicefrac{1}{2}$]{TL}}    & Salz\\
        etwas                             & Zitronenschale\\
        {\unit[1]{P.}}                    & Mohnback\\
                                          & \\
        Guss:                             & \\
                                          & Hagelzucker\\
                                          & Sahne
    }
    \preparation{
        \step Alle Teigzutaten 15 Minuten lang im Brotbackautomaten kneten lassen und 1 Stunde gehen lassen.
        \step Teig auf einem Backblech zum Rechteck ausrollen; längs halbieren.
        \step Mohnback auf dem Teig verteilen, längs aufrollen und beide miteinander verdrehen.
        \step Auf dem Bakblech 40 Minuten lang gehen lassen; mit Sahne bestreichen und Hagelzucker darauf streuen.
        \step Dann bei {\unit[170]{\celsius}} im vorgeheizten ca. 35 Minuten backen.
    }
\end{recipe}
\end{document}
