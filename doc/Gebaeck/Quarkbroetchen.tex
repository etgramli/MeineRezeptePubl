\makeatletter
\def\input@path{{../../}}
\makeatother
\providecommand{\main}{../../}
\documentclass[MeineRezepte.tex]{subfiles}
\begin{document}
\begin{recipe}[
    bakingtime = {\unit[15-20]{Min}},
    bakingtemperature = {\unit[180]{\celsius}},
    source = Lili
]{Quarkbrötchen}\index{Quarkbroetchen@Quarkbrötchen}
    \graph{
        big=Quarkbroetchen_0
    }
    \ingredients{
        2                    & Eier\\
        {\unit[250]{g}}      & Quark\\
        {\unit[1]{P.}}       & Vanille-Zucker\\
        {\unit[1]{Prise}}    & Salz\\
        {\unit[3]{EL}}       & Zucker\\
        {\unit[200]{g}}      & Mehl\\
        {\unit[1]{P.}}       & Backpulver\\
                             & Zitronenaroma\\
                             & Rosinen\\
                             & Nüsse, gehackt
    }
    \preparation{
        \step Eier, Quark, Vanille-Zucker, Salz, Zucker und Zitronenaroma gut vermischen.
        \step Das Mehl und das Backpulver mischen und zu der Quarkmasse geben.
        \step Mit 2 Esslöffeln 9-12 Teighaufen auf ein Backblech setzen und mit Nüssen bestreuen.
        \step Bei {\unit[180]{\celsius}} {\unit[15-20]{Minuten}} im Backofen backen.
    }
    \hint{%
        Ist am nächsten Tag leider sehr trocken\newline
        Geht auch gut als Muffins; geht auch mit etwas weniger Mehl
    }
\end{recipe}
\end{document}
