\makeatletter
\def\input@path{{../../}}
\makeatother
\providecommand{\main}{../../}
\documentclass[MeineRezepte.tex]{subfiles}
\begin{document}
\begin{recipe}[
    preparationtime = {\unit[45]{Min}},
    bakingtime = {\unit[50]{Min}},
    bakingtemperature = {\unit[220]{\celsius}},
    portion = {\portion[Laib]{1}},
    source = Lili
]{Roggen-Sauerteig-Brot}\index{Roggen-Sauerteig-Brot}\index{Brot!Roggen-Sauerteig-Brot}
    \graph{
        big=RoggenSauerteigBrot_0
    }
    \ingredients{
        {\unit[180]{g}}                   & Roggenkörner\\
        {\unit[200]{g}}                   & Weizenmehl 405\\
        {\unit[1]{TL}}                    & Salz\\
        {\unit[3]{TL}}                    & Sauerteigpulver\\
        {\unit[$\nicefrac{1}{2}$]{TL}}    & Brotgewürz\\
        {\unit[15]{g}}                    & Hefe\\
        {\unit[130]{g}}                   & Malzbier oder Bier\\
        {\unit[130]{g}}                   & Wasser, lauwarm\\
                                          &
    }
    \preparation{
        \step Die Roggenkörner in dem Mixtopf 10 Sekunden bei Stufe 10 malen.
        \step Die Restlichen Zutaten dazu geben und circa 4 Minuten bei Teigstufe vermischen.
        \step Einen Römertopf oder Kasserolle (mit Deckel) einfetten und den Boden mit Haferflocken bestreuen.
        \step Den Teig formen, in das Gefäß legen und einschneiden.
        \step Mit Mehl bestreuen und $\nicefrac{1}{2}$ Stunde mit Deckel gehen lassen.
        \step Dann den Teig mit geschlossenem Deckel im kalten Ofen {\unit[50]{Min}} bei {\unit[220]{\celsius}} backen.
    }
\end{recipe}
\end{document}
