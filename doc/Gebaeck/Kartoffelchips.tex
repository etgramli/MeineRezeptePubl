\makeatletter
\def\input@path{{../../}}
\makeatother
\providecommand{\main}{../../}
\documentclass[MeineRezepte.tex]{subfiles}
\begin{document}
\begin{recipe}[
    preparationtime = {\unit[50]{Min}},
    bakingtime = {\unit[30]{Min}},
    bakingtemperature = {\unit[180]{\celsius}},
    portion = {\portion[Bleche]{2}},
    source = Etienne
]{Kartoffelchips}\index{Kartoffelchips}
    \graph{
        small=Kartoffelchips_1,
        big=Kartoffelchips_0
    }
    \ingredients{
        4                     & Kartoffeln, groß\\
        {\unit[4-6]{EL}}      & Olivenöl\\
        {\unit[2]{TL}}        & Paprikapulver\\
        {\unit[2]{Prisen}}    & Salz
    }
    \preparation{
        \step Kartoffeln gut waschen, nach Belieben schälen und mit einem Hobel in gleichmäßige Scheiben schneiden.
        \step Die Scheiben 20-30 Minuten wässern; inzwischen das Öl mit dem Paprikapulver vermischen.
        \step Die Kartoffelscheiben gut abtrocknen und mit Öl bestreichen, sie müssen nicht komplett bedeckt sein.
        \step Im vorgeheizten Backofen ca. {\unit[30]{Min}} bei {\unit[180]{\celsius}} backen, bis die gewünschte Bräune erreicht ist; den Ofen regelmäßig öffnen, damit die Feuchtigkeit entweicht.
        \step Die Chips auf Küchenkrepp legen und etwas salzen.
    }
    \hint{%
        Lecker sind auch Minz-Chips mit 10 Minz-Blättern statt Paprikapulver.
    }
\end{recipe}
\end{document}
