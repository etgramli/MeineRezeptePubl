\makeatletter
\def\input@path{{../../}}
\makeatother
\providecommand{\main}{../../}
\documentclass[MeineRezepte.tex]{subfiles}
\begin{document}
\begin{recipe}[
    bakingtime = {\unit[10]{Min}},
    bakingtemperature = {\unit[200]{\celsius}},
    portion = {\portion[Bleche]{3}},
    source = Lili
]{Zitronenplätzchen}\index{Zitronenplaetzchen@Zitronenplätzchen}
    \graph{
        big=Zitronenplaetzchen_0
    }
    \ingredients{
        6                    & Eigelb\\
        {\unit[250]{g}}      & Butter\\
        {\unit[200]{g}}      & Zucker\\
        {\unit[1]{Prise}}    & Salz\\
        {\unit[500]{g}}      & Mehl\\
        {\unit[1]{TL}}       & Backpulver\\
        {\unit[2]{EL}}       & Zitronensaft\\
                             & Zitronenschale\\
                             & \\
        Guss:                & \\
        {\unit[1]{EL}}       & Puderzucker\\
        {\unit[1]{EL}}       & Zitronensaft\\
        {\unit[1]{EL}}       & Wasser, heiß
    }
    \preparation{
        \step Butter, Zucker, Salz und Eigelb schaumig rühren; die restlichen Zutaten darunter arbeiten.
        \step Mindestens 1 Stunde kalt stellen; mit Mehl ausrollen, ausstechen und bei {\unit[200]{\celsius}} ca. 10 Minuten backen.
        \step Die Zutaten für den Guss verrühren und auf den Plätzchen verteilen.
    }
    \hint{%
        Man kann daraus auch Orangenplätzchen machen.\newline
        Schmecken wie Butterplätzchen.
    }
\end{recipe}
\end{document}
