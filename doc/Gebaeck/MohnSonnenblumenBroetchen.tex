\makeatletter
\def\input@path{{../../}}
\makeatother
\providecommand{\main}{../../}
\documentclass[MeineRezepte.tex]{subfiles}
\begin{document}
\begin{recipe}[
    preparationtime = {\unit[20]{Min}},
    bakingtime = {\unit[35]{Min}},
    bakingtemperature = {\unit[160]{\celsius}},
    portion = {\portion[Stück]{4}},
    source = Lili
]{Mohn-Sonnenblumen-Brötchen}\index{Mohn-Sonnenblumen-Broetchen@Mohn-Sonnenblumen-Brötchen}\index{Brot!Mohn-Sonnenblumen-Broetchen@Mohn-Sonnenblumen-Brötchen}
    \graph{
        big=MohnSonnenblumenBroetchen_0
    }
    \ingredients{
        2                                 & Eier\\
        {\unit[250]{g}}                   & Quark (\unit[40]{\%})\\
        {\unit[90]{g}}                    & Mandeln, gemahlen\\
        {\unit[50]{g}}                    & Sonnenblumenkerne\\
        {\unit[30]{g}}                    & Mohn / Leinsamen\\
        {\unit[20]{g}}                    & Flohsamenschalen\\
        {\unit[10]{g}}                    & Kokosmehl\\
        {\unit[$\nicefrac{1}{2}$]{TL}}    & Salz\\
        {\unit[$\nicefrac{1}{2}$]{P.}}    & Backpulver
    }
    \preparation{
        \step Eier und Quark mit dem Salz zu einer cremigen Masse verquirlen.
        \step Trockene Zutaten in einer Schüssel mischen und mit der Masse zu gleichmäßigem Teig verquirlen.
        \step Teig 10 Minuten gehen lassen und danach nochmal gut durchrühren.
        \step Vier runde Brötchen formen und auf ein mit Backpapier ausgelegtes Backblech legen.
        \step Im vorgeheizten Backofen {\unit[35]{Minuten}} bei {\unit[160]{\celsius}} backen.
    }
    \hint{%
        Wenn die Masse zu pappig ist, kann man Weizen- oder Haferkleie dazu geben.
    }
\end{recipe}
\end{document}
