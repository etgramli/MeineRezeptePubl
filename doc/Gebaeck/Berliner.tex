\makeatletter
\def\input@path{{../../}}
\makeatother
\providecommand{\main}{../../}
\documentclass[MeineRezepte.tex]{subfiles}
\begin{document}
\begin{recipe}[
    preparationtime = {\unit[15]{Min}},
    bakingtime = {\unit[6]{Min}},
    portion = {\portion[Stück]{10}},
    source = Lili
]{Berliner}\index{Berliner}
    \graph{
        small=Berliner_1,
        big=Berliner_0
    }
    \ingredients{
        2                                   & Eier\\
        {\unit[100]{ml}}                    & Milch\\
        {\unit[100]{g}}                     & Butter\\
        {\unit[500]{g}}                     & Mehl\\
        {\unit[1]{P.}}                      & Trockenhefe\\
        {\unit[40]{g}}                      & Zucker\\
        {\unit[2]{EL}}                      & Vanille-Zucker\\
        {\unit[1]{Prise}}                   & Salz\\
        {\unit[$\nicefrac{1}{4}$]{TL}}      & Zitronenaroma\\
        {\unit[$\nicefrac{1}{2}$]{Glas}}    & Marmelade\\
        etwas                               & Zitronensaft\\
                                            & Puderzucker
    }
    \preparation{
        \step Die trockenen Zutaten mischen; Eier, geschmolzene Butter und kalte Milch dazu geben.
        \step In der Brotmaschine 10~Minuten kneten lassen und eine Stunde gehen lassen.
        \step Mit Mehl dick ausrollen und mit einem Glas ausstechen und nochmals 40~Minuten gehen lassen.
        \step Im vorgeheizten Öl bei Stufe 3 von 9 pro Seite 2-3~Minuten ausbacken.
        \step Mit Puderzucker bedecken, etwas abkühlen lassen und füllen.
    }
    \hint{%
        Etwas weniger Milch!
    }
\end{recipe}
\end{document}
